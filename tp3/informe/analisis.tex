A continuación se presenta un análisis comparativo de los tres métodos implementados.
Cabe mencionar que, dado que la experimentación requería que ambas imágenes (original y modificada) tengan el mismo tamaño para poder realizar un análisis cuantitativo de los algoritmos mediante las medidas de comparación que se mencionaron en la introducción, decidimos achicar la imagen original mediante un script que creamos para luego agrandarla mediante nuestros métodos y poder comparar los resultados obtenidos con la imagen original.
\\
El recortador de imágenes, que se encuentra en la carpeta $Recortador de imagenes$, toma una imagen como input la cual se quiere achicar para poder comparar con si misma en tamaño original una vez aplicado el algoritmo de zoom y un entero $k$ que es el factor de achicamiento, que debe ser el mismo que usemos a la hora de aplicar zoom sobre esta imagen achicada. Este script lo hicimos utilizando $OpenCV$, de modo similar a las funciones utilizadas para hacer zoom en la experimentación base.
\\ 
Este programa $recortador.cpp$ tiene un funcionamiento sencillo.
Lo primero que hacemos es contar cuentas filas y columnas va a tener la nueva imagen achicada, esto depende del tamaño del $k$, vamos contando de a k partes, ya que la idea es sacar las k filas y columnas intermedias para poder volver a hacer lo mismo cuando vayamos a agrandar la imagen. Luego creamos una imagen nueva con ese tamaño y vamos copiando los píxeles de las posiciones (i*k, j*k), salteando de a $k$ posiciones y así evitarnos las $k$ columnas y filas para después cuando volvemos a agrandar la imagen, poder llenar esas mismas posiciones faltantes como estaban antes. Así evitamos sesgar los experimentos, ya que achicamos la imagen sin ningún criterio particular, o sea, el criterio para achicar no tiene ninguna relación con el criterio para agrandar la imagen con cualquiera de los 3 métodos. Por lo tanto, ninguno de los 3 métodos va a tener un beneficio dependiendo de como se achicó la imagen.   

\subsection{Correctitud de la implementación}

El primer método para comprobar rápidamente la correctitud de la implementación de nuestros algoritmos fue sencillamente comparar \"a ojo\" las imágenes resultantes con la original, observando el nivel de detalle obtenido. Al comienzo, teniendo errores de implementación, notamos de este modo que la implementación necesitaba mejoras.

Luego, para obtener mayor rigor, procedimos a comparar nuestros algoritmos con aquellos que vienen por defecto en opencv. Consideramos que estos algoritmos son lo suficientemente fiables como para tomarlos como punto de referencia. Tomamos una imagen:
\begin{figure}[H]
    \centering
\subfloat[.]{{\includegraphics[width=7cm]{fotos/opencvvsus/orig.png} }}
\caption{Imagen Original}
\end{figure}
Y la reescalamos en la mismas dimenciones tanto con nuestros algoritmos como con los implementados en opencv.

\begin{figure}[H]
    \centering
    \subfloat[Nosotros]{{\includegraphics[width=7cm]{fotos/opencvvsus/1us.png} }}
    \qquad
    \subfloat[OpenCV]{{\includegraphics[width=7cm]{fotos/opencvvsus/1open.png} }}
    \caption{Comparacón de correctitud contra opencv: Vecinos Mas Cercanos}
\end{figure}

\begin{figure}[H]
    \centering
    \subfloat[Nosotros]{{\includegraphics[width=7cm]{fotos/opencvvsus/2us.png} }}
    \qquad
    \subfloat[OpenCV]{{\includegraphics[width=7cm]{fotos/opencvvsus/2open.png} }}
    \caption{Comparacón de correctitud contra opencv: Bilineal}
\end{figure}

\begin{figure}[H]
    \centering
    \subfloat[Nosotros]{{\includegraphics[width=7cm]{fotos/opencvvsus/3us.png} }}
    \qquad
    \subfloat[OpenCV]{{\includegraphics[width=7cm]{fotos/opencvvsus/3open.png} }}
    \caption{Comparacón de correctitud contra opencv: Bicubico con ventanas de $4\times4$ pixels}
\end{figure}

Como puede verse, nuestros algoritmos arrojan resultados muy parecidos a opencv. Por lo que consideramos que su implementacion es correcta.

\subsection{Ventana óptima para método de Splines}
Nuestro primer análisis se encargará de encontrar un valor óptimo para la cantidad de las ventanas utilizadas en el método de splines.
Para dicho fin, elegimos correr varias instancias del método con valores de $K$ crecientes para distintos valores de ventana (4, 8, y 16 para poder comparar los resultados). Se presentan entonces los resultados obtenidos. Vale la pena destacar que no se muestra información respecto al PSNR debido a que presentaba exactamente el mismo comportamiento y no ofrecia informacion extra alguna.
\\
Como habiamos presupuesto en la introducción, agrandar el tamaño de la ventana solo hace que se tengan en cuenta valores para el punto que se quiere calcular que no depende directamente de este. 

\begin{center}
\includegraphics[scale=0.50]{imagenes/VK2.png}
\end{center}

Como puede verse a continuación, agrandar el tamaño de la ventana deja de presentar benificio alguno para valores mas altos de $K$ porque los resultados se veulven constantes:

\begin{center}
\includegraphics[scale=0.50]{imagenes/VK4.png}
\includegraphics[scale=0.50]{imagenes/VK6.png}
\includegraphics[scale=0.50]{imagenes/VK10.png}
\end{center}

A partir de este punto, el algoritmo de splines utiliza una ventana de tamaño cuatro, dado que es la que presenta mejores resultados.
De todos formas, no es cierto que siempre sea preferible una ventana más chica a una que incluya mas puntos, porque en problemas donde se quieren calcular por ejemplo trayectorias, es deseable considerar mas puntos para tener mas información.

\subsection{Analisis de los metodos}
\subsubsection{Analisis de los metodos para imagenes con simbolos alfanumericos}
En esta sección analizaremos los tres algoritmos sobre imagenes con simbolos alfanumericos. Para ellos usamos la imagen mostrada mas abajo para la cual aplicaremos los tres metodos con diferentes ks. La caracteristica principal a testear sera la capacidad de dicernimiento de estos simbolos despues de aplicados los metodos de zoom.

\begin{figure}[H]
\centering
\includegraphics[scale=0.50]{fotos/alfanum/orig.png}
\end{figure}

Primero realizamos las pruebas con el valor minimo de $k$ ($k=1$), obteniendo los siguientes resultados:


\begin{figure}[H]
    \centering
    \subfloat[Metodo 1]{{\includegraphics[width=5cm]{fotos/alfanum/k1_1.jpg} }}
    \qquad
    \subfloat[Metodo 2]{{\includegraphics[width=5cm]{fotos/alfanum/k1_2.jpg} }}
    \qquad
    \subfloat[Metodo 3]{{\includegraphics[width=5cm]{fotos/alfanum/k1_3.jpg} }}
    \caption{Comparacón de metodos para $k = 1$}
    \label{fig:example}
\end{figure}

Como podemos ver, las tres imagenes introducen artifacts (errores visuales) que todavia no desmejoran la imagen a un nivel en el que sea imposible su comprension, por lo menos en los digitos mas externos (distinto para los numeros internos de la imagen que, debido a su tamaño inicial, ya son casi inperceptibles con este $k$ minimo). Notese como el metodo 2, como consecuencia de la nivelacion entre los valores de alto contraste del dibujo y su fondo blanco, empieza a introducir una leve 'niebla gris' alrededor de las imagenes. La misma sutiacion se plantea en el metodo tres, pero con la diferencia de que el difuminado introducido es mucho menos visible.


Observemos los resultados para un $k$ un poco mas elevado ($k=2$):

\begin{figure}[H]
    \centering
    \subfloat[Metodo 1]{{\includegraphics[width=5cm]{fotos/alfanum/k2_1.jpg} }}
    \qquad
    \subfloat[Metodo 2]{{\includegraphics[width=5cm]{fotos/alfanum/k2_2.jpg} }}
    \qquad
    \subfloat[Metodo 3]{{\includegraphics[width=5cm]{fotos/alfanum/k2_3.jpg} }}
    \caption{Comparacón de metodos para $k = 2$}
    \label{fig:example}
\end{figure}

En esta ocacion, el comportamiento sigue los lineamientos generales del caso anterior, con la salvedad de que ninguna de las tres imagenes ya es comprensible. Vemos como la 'niebla' comentada en el caso anterior avanzo rapido en el metodo dos, para casi difuminar la imagen por completo. En el caso del metodo de splines (el tercero)se puede empezar a ver un pequeño sombreado alrededor de los bordes de los elementos en la imagen, pero a diferencia del metodo dos, esta solo se extiende a las cercanias y no avanza por toda la imagen.


Por ultimo, presentamos los resultados para $k=4$:

\begin{figure}[H]
    \centering
    \subfloat[Metodo 1]{{\includegraphics[width=5cm]{fotos/alfanum/k5_1.jpg} }}
    \qquad
    \subfloat[Metodo 2]{{\includegraphics[width=5cm]{fotos/alfanum/k5_2.jpg} }}
    \qquad
    \subfloat[Metodo 3]{{\includegraphics[width=5cm]{fotos/alfanum/k5_3.jpg} }}
    \caption{Comparacón de metodos para $k = 4$}
    \label{fig:example}
\end{figure}

Como era de esperarse las tres imagenes resultantes ya perdieron comprension en su totalidad. Ademas, el segundo y tercer metodo, presentan una alta cantidad de ruido por el difuminado producido respecto de la imagen original.
Queda entonces a la vista una caracteristica que no estabamos considerando hasta entonces en nuestro analisis. El metodo de los vecinos puede llegar a ofrecer resultados favorables si se cumplen algunas caracteristicas deseables (nuestra intuición preveia que este metodo seria superado por los anteriores en cualquier situacion) como en este caso. El alto contraste entre las imagenes, hace que en los metodos que introducen cierta correlación o suavizado entre pixeles se genere un sombreado que hace mas borrosas las imagenes. En contra de nuestros pronosticos, el metodo de los vecinos podria ser un excelente candidato en estos casos.


\subsubsection{Analisis de los metodos para paisajes}

En esta sección analizamos como se comportan los metodos para fotos de paisajes. Consideramos una imagen que no presente grandes contrastres como en el analisis anterior y en la que se puedan analizar tanto detalles puntuales (la definicion de las ramas del arbol) asi como aquellos mucho mas definidos (piedras y ramas que cubren un gran porcentaje de la imagen). Tomamos la siguiente foto de ejemplo:

\begin{figure}[H]
\centering
\includegraphics[scale=0.50]{fotos/paisaje/orig.png}
\end{figure}

Primero lo hacemos para $k=1$, se obtiene esto:

\begin{figure}[H]
    \centering
    \subfloat[Metodo 1]{{\includegraphics[width=5cm]{fotos/paisaje/k1_1.png} }}
    \qquad
    \subfloat[Metodo 2]{{\includegraphics[width=5cm]{fotos/paisaje/k1_2.png} }}
    \qquad
    \subfloat[Metodo 3]{{\includegraphics[width=5cm]{fotos/paisaje/k1_3.png} }}
    \caption{Comparacón de metodos para $k = 1$}
    \label{fig:example}
\end{figure}

En este primer ejemplo, con un $k$ minimo, ninguna de las tres imagenes presenta una calidad demasiado desmejorada. En particular nos sorprendio ver la cantidad de ruido introducida por el tercer metodo, que visualmente esta mas cerca al metodo de los vecinos (el cual se perfilaba como el de peor rendimiento de los tres y termino en segundo lugar) que al metodo dos (de el cual, de hecho, suponiamos era una mejora).
Confirmando nuestra apreciacion, el $ECM$ y $PSNR$ obtenido para cada uno de los metodos fue el siguiente:
\begin{itemize}
 \item Metodo 1: ECM de $315.723$ y PSNR de $23.1377$.
 \item Metodo 2: ECM de $115.114$ y PSNR de $27.5195$.
 \item Metodo 3: ECM de $332.629$ y PSNR de $22.9112$
\end{itemize}

Ahora lo hacemos para $k=2$, se obtiene esto:

\begin{figure}[H]
    \centering
    \subfloat[Metodo 1]{{\includegraphics[width=5cm]{fotos/paisaje/k2_1.png} }}
    \qquad
    \subfloat[Metodo 2]{{\includegraphics[width=5cm]{fotos/paisaje/k2_2.png} }}
    \qquad
    \subfloat[Metodo 3]{{\includegraphics[width=5cm]{fotos/paisaje/k2_3.png} }}
    \caption{Comparacón de metodos para $k = 2$}
    \label{fig:example}%
\end{figure}

En este segundo caso, el cambio en el $k$ es minimo. Sin embargo, podemos apreciar como el metodo de los vecinos y el de splines (metodo 1 y 3 respectivamente) empiezan a introducir una gran cantidad de ruido. En el caso del metodo dos, a diferencia de lo ocurrido durante el analisis de caracteres alfanumericos, el suavizado que se produce en la imagen si ayuda a que esta se mantenga entendible y el difuminado termina favoreciendo a la comprension de la misma (esto no ocurria en los caracteres alfanumericos, porque este mismo suavizado termina oscureciendola y quitandole claridad).
Una vez mas, los valores de error cuadrático medio y la relación signal/noise apoyan el analisis realizado e incluso demuestran que la diferencia introducida en el ECM se duplica entre cada metodo.
\begin{itemize}
 \item Metodo 1: ECM de $724.517$ y PSNR de $19.5303$.
 \item Metodo 2: ECM de $293.755$ y PSNR de $23.4509$.
 \item Metodo 3: ECM de $444.18$ y PSNR de $21.6552$
\end{itemize}


Ahora lo hacemos para $k=3$, se obtiene esto:
\begin{figure}[H]
    \centering
    \subfloat[Metodo 1]{{\includegraphics[width=5cm]{fotos/paisaje/k3_1.png} }}%
    \qquad
    \subfloat[Metodo 2]{{\includegraphics[width=5cm]{fotos/paisaje/k3_2.png} }}%
    \qquad
    \subfloat[Metodo 3]{{\includegraphics[width=5cm]{fotos/paisaje/k3_3.png} }}%
    \caption{Comparacón de metodos para $k = 3$}%
    \label{fig:example}%
\end{figure}

En este ultimo ejemplo, podemos ver como el aumento minimo del valor de $k$ (tan solo sumando uno mas) produce niveles altisimos de perdida de definicion en las tres imagenes. Contraria a nuestra intuición, el metodo dos sigue ofreciendo un mejor desempeño en este caso, incluso contra el metodo de Splines. El metodo de los vecinos (metodo uno), que no presenta ningun suavizado, queda rapidamente relegada al ultimo lugar en cuanto a la cantidad de ruido (Notese como ya es dificil diferenciar las zonas de mayor definicion como las ramas del arbol e incluso empieza a estar comprometida nuestra capacidad de diferenciar donde empieza la piedra que se encuentra en la zona media izquierda y donde lo hace la superficie del piso). En el metodo dos podemos confirmar que, como en el caso anterior, a cambio de introducir cierto sombreado en la imagen se consigue la mejor definicion colocandolo nuevamente como el de mejor desempeño.
Una vez mas, los valores de $ECM$ y $PSNR$ son, como era de esperarse, los siguientes:
\begin{itemize}
 \item Metodo 1: ECM de $1099.83$ y PSNR de $17.7175$.
 \item Metodo 2: ECM de $400.854$ y PSNR de $22.1009$.
 \item Metodo 3: ECM de $493.992$ y PSNR de $21.1936$
\end{itemize}
Notese como los metodos dos y tres se presentan una pequeña diferencia comparandola con la cantidad de error introducida por el primer metodo.

Testeamos con un valor extremo de $k$ para ver cuales son los resultados, para $k=10$ se obtiene:

\begin{figure}[H]
    \centering
    \subfloat[Metodo 1]{{\includegraphics[width=5cm]{fotos/paisaje/k10_1.png} }}%
    \qquad
    \subfloat[Metodo 2]{{\includegraphics[width=5cm]{fotos/paisaje/k10_2.png} }}%
    \qquad
    \subfloat[Metodo 3]{{\includegraphics[width=5cm]{fotos/paisaje/k10_3.png} }}%
    \caption{Comparacón de metodos para $k = 3$}%
    \label{fig:example}%
\end{figure}

%ECM        PSNR
%2976.63    13.3936
%1266.65    17.1042
%1454.78    16.5028


\subsection{Analisis de los metodos para rostros}


En esta ultima sección analizamos como se comportan los metodos para fotos de rostros. Centraremos nuestro analisis en el comportamiento de los tres algoritmos frente a rasgos particulares del rostro para poder valorar la calidad de los mismos. Tomamos la siguiente foto:

\begin{figure}[H]
\centering
\includegraphics[scale=0.50]{fotos/rostro/orig.png}
\end{figure}

Presentamos el mismo analisis que en las imagenes anteriores ($k=1$, $k=2$ y $k=3$)

\begin{figure}[H]
    \centering
    \subfloat[Metodo 1]{{\includegraphics[width=5cm]{fotos/rostro/k1_1.png} }}%
    \qquad
    \subfloat[Metodo 2]{{\includegraphics[width=5cm]{fotos/rostro/k1_2.png} }}%
    \qquad
    \subfloat[Metodo 3]{{\includegraphics[width=5cm]{fotos/rostro/k1_3.png} }}%
    \caption{Comparacón de metodos para $k = 1$}%
    \label{fig:example}%
\end{figure}

\begin{itemize}
 \item Metodo 1: ECM de $95.6087$ y PSNR de $28.3258$.
 \item Metodo 2: ECM de $32.6121$ y PSNR de $32.997$.
 \item Metodo 3: ECM de $105.23$ y PSNR de $27.9094$
\end{itemize}

\begin{figure}[H]
    \centering
    \subfloat[Metodo 1]{{\includegraphics[width=5cm]{fotos/rostro/k2_1.png} }}%
    \qquad
    \subfloat[Metodo 2]{{\includegraphics[width=5cm]{fotos/rostro/k2_2.png} }}%
    \qquad
    \subfloat[Metodo 3]{{\includegraphics[width=5cm]{fotos/rostro/k2_3.png} }}%
    \caption{Comparacón de metodos para $k = 2$}%
    \label{fig:example}%
\end{figure}

\begin{itemize}
 \item Metodo 1: ECM de $235.245$ y PSNR de $24.4156$.
 \item Metodo 2: ECM de $87.1968$ y PSNR de $28.7258$.
 \item Metodo 3: ECM de $150.451$ y PSNR de $26.3569$
\end{itemize}

\begin{figure}[H]
    \centering
    \subfloat[Metodo 1]{{\includegraphics[width=5cm]{fotos/rostro/k3_1.png} }}
    \qquad
    \subfloat[Metodo 2]{{\includegraphics[width=5cm]{fotos/rostro/k3_2.png} }}
    \qquad
    \subfloat[Metodo 3]{{\includegraphics[width=5cm]{fotos/rostro/k3_3.png} }}
    \caption{Comparacón de metodos para $k = 3$}
    \label{fig:example}
\end{figure}


\begin{itemize}
 \item Metodo 1: ECM de $389.559$ y PSNR de $22.2251$.
 \item Metodo 2: ECM de $113.193$ y PSNR de $27.5926$.
 \item Metodo 3: ECM de $159.421$ y PSNR de $26.1054$
\end{itemize}

Este analisis vuelve a repetir el mismo comportamiento que vimos en el caso de las imagenes de paisajes en cuanto a la relacion entre el error introducido por cada uno de los metodos (aunque con distintos valores, la relacion en el fondo es la misma). Sin embargo, y como una apreciacion totalmente subjetiva, consideramos que el metodo tres (aproximacion por splines), conserva mucho mejor las caracteristicas del rostro de la imagen que los otros dos; el primer caso debido a la excesiva 'pixelacion' de la imagen, y el caso tres debido a que, debido al suavizado general de toda la imagen que ya mencionamos en ocasiones anteriores, difumina demasiado los detalles menores del rostro de la imagen.

Testeamos con un valor extremo de $k$ para ver cuales son los resultados, para $k=10$ se obtiene:

\begin{figure}[H]
    \centering
    \subfloat[Metodo 1]{{\includegraphics[width=5cm]{fotos/rostro/k10_1.png} }}%
    \qquad
    \subfloat[Metodo 2]{{\includegraphics[width=5cm]{fotos/rostro/k10_2.png} }}%
    \qquad
    \subfloat[Metodo 3]{{\includegraphics[width=5cm]{fotos/rostro/k10_3.png} }}%
    \caption{Comparacón de metodos para $k = 3$}%
    \label{fig:example}%
\end{figure}
%metodo 1
%ECM        PSNR
%1563.05    16.1911
%metodo 2
%684.334    19.7781
%metodo 3
%639.892    20.0697

\subsubsection{Conclusiones del analisis de los metodos}
Como pudimos apreciar a lo largo de los distintos analisis realizados, no existe un metodo que sea claramente un ganador.
Si bien es cierto que el metodo dos es el que introduce la menor cantidad de errores, una apreciacion mas 'subjetiva' y alejada de los numeros nos deja las siguientes sensaciones:
\begin{enumerate}
 \item El metodo de los vecinos puede llegar a mostrar mejores resultados en situaciones de imagenes que consisten en elementos frente a un fondo uniforme con un alto contraste debido a que al no realizar ninguna interpolación entre los pixeles evita introducir sombreados innecesarios.
 \item El metodo bilineal, a pesar de ser el mejor en cuanto a errores y la cantidad de ruido introducida en las imagenes, no siempre es considerado el metodo optimo segun nuestro analisis. En situaciones donde la imagen presenta regiones de alta cantidad de detalles (como puede ser el rostro de una persona en el caso analizado), genera un sombreado que hace perder definicion en estas zonas. Situacion que no ocurre para el metodo de Splines. Sin embargo, en imagenes donde no hay regiones de grandes detalles (como puede ser un paisaje) si lo consideramos el optimo.
 \item El metodo de interpolacion de Splines, que presenta valores cercanos al mejor cantidato (el metodo Bilineal) en la mayoria de los casos, tiene la ventaja de que, al introducir menor cantidad de 'sombreado' que este ultimo, conserva mejor aquellos detalles mencionados en donde el metodo Bilineal perdia claridad.
\end{enumerate}


% \subsection{Análisis de los metodos}
%Yo diría que esto no va mas

% Empleamos un análisis incremental respecto al valor de los pixeles intermedios introducidos (el valor de $k$) para poder analizar los distintos métodos de forma escalonada y presentar conclusiones mucho más claras. Dado que nuestros algoritmos solo funcionan cuando los valores de las imágenes son divisibles por $k$, no deben asumirse una correlación entre los distintos valores de este debido a que las imágenes a analizar no siempre pudieron ser las mismas.

% \subsubsection{K = 2}
% Nuestro primer análisis se concentra en el valor mínimo de $k$ para el cual esperabamos que el comportamiento de los tres métodos se mantenga bastante estable. Nuestra intuición proviene de la idea de que todos ellos ofrecian una perdida en la calidad de la imagen bastanta pequeña en relación al zoom pedido.
% Como podemos apreciar en los gráficos a continuación, nuestra intuición se corresponde con los valores de $PSNR$, donde los tres métodos se comportan relativamente iguales, sin embargo, nos sorprende ver que para el error cuadrático medio (y para $PSNR$ también, pero en menor medida) la técnica de interpolación Bilineal obtuvo resultados muy destacables (de hecho, casi constantes), incluso frente a la técnica de Splines que esperabamos siempre tenga un mejor rendimiento. La conclusión respecto de este fenómeno se explica al final del artículo.

% \begin{center}
% \includegraphics[scale=0.50]{imagenes/K2PSNR.png}
% \includegraphics[scale=0.50]{imagenes/K2ECM.png}
% \end{center}

% \subsubsection{K = 4}
% Para este segundo caso es cierto que, como esperabamos, el error cuadrático medio del método de los vecinos se dispara rápidamente mientras que los de interpolación bilineal y splines se mantienen prácticamente iguales. Lo mismo sucede para los valores de PSNR, siendo los del método de vecinos los únicos que disminuyen con una diferencia de casi 5 puntos.

% \begin{center}
% \includegraphics[scale=0.50]{imagenes/K4PSNR.png}
% \includegraphics[scale=0.50]{imagenes/K4ECM.png}
% \end{center}

% \subsubsection{K = 6}
% Entrando en valores de $k$ mucho más elevados, nuestro análisis empezó a dejar de coincidir con lo que creiamos en un primer momento serían los resultados finales, debido a que el método de Splines no logró sacar una diferencia notoria frente al de interpolación Bilineal, sino que incluso ambos métodos se mantuvieron prácticamente constantes.  Al momento de obtener los resultados nos llamaron poderosamente la atención los valores de ECM obtenidos para las tres primeras imágenes, pero luego de hacer un análisis en conjunto de estas, llegamos a la conclusión de que la suba desmesurada en estos valores se debe a la elevada variabilidad de los contrastes en la escena (el interior de un hogar muy decorado, la foto aérea de un barrio, etc). 

% \begin{center}
% \includegraphics[scale=0.50]{imagenes/K6PSNR.png}
% \includegraphics[scale=0.50]{imagenes/K6ECM.png}
% \end{center}

% \subsubsection{K = 10}
% Por último, para valores que ya se consideran altos de $k$ el método de interpolación Bilineal todavía sigue desempeñandose igual o incluso a veces levemente mejor que el de Splines. Esto no solo contradice nuestra intuición, sino que debido a la performance de ambos, estos resultados colocan al método de interpolación como el más apto en relación benenficio/tiempo, muy por encima del de Splines (ambos ya muy por encima del método de vecinos a esta altura).

% \begin{center}
% \includegraphics[scale=0.50]{imagenes/K10PSNR.png}
% \includegraphics[scale=0.50]{imagenes/K10ECM.png}
% \end{center}

\subsection{Análisis de tiempos}
El análisis de tiempo, a diferencia del de los métodos, no ofreció ninguna respuesta que no hayamos podido intuir durante la codificación de los algoritmos. Es claro que a medida que el método se perfecciona en la búsqueda de resultados más suaves, también aumenta el tiempo necesario de cálculo.

\begin{center}
\includegraphics[scale=0.50]{imagenes/K2T1.png}
\includegraphics[scale=0.50]{imagenes/K2T2.png}
\includegraphics[scale=0.50]{imagenes/K4T.png}
\includegraphics[scale=0.50]{imagenes/K6T.png}
\includegraphics[scale=0.50]{imagenes/K10T.png}
\end{center}

