En el presente trabajo práctico nos encargaremos de analizar el problema de la interpolación de polinomios mediante distintos métodos.
Para ello se nos ofrece como marco la necesidad de realizar zoom a distintas imágenes, con el fin de crear un prototipo que permita decidir en tiempo
real si una pelota entró o no dentro de un arco de fútbol.

Nuestro objetivo entonces es, dada una imagen de $n x m$ pixeles de tamaño original y un número natural $k$ que denota la cantidad de filas y
columnas que se quieren agregar entre cada pixel, encontrar la forma mas óptima de rellenar estos valores.

Presentaremos entonces tres técnicas a detallar con sus respectivas ventajas y desventajas:
\begin{enumerate}
 \item Vecinos
 \item Interpolacion bilineal
 \item Interpolacion por splines
\end{enumerate}

Además, con el fin de poder realizar un análisis cuantitavo sobre cada método, consideraremos dos medidas que comparan las imágenes
originales contra sus transformadas ofreciendo una noción de error o ruido:
\begin{enumerate}
 \item Error cuadrático medio (ECM)
 \item Peak to signal noise (PSNR)
\end{enumerate}

Por último, nuestra visión preliminar del problema nos indica que si bien es posible que los tres algoritmos retornen buenos resultados
para valores de $k$ relativamente bajos ($k = 1$ y quizas hasta $2$) y la versión de vecinos sea mucho más eficiente temporalmente,
es probable que esta deje de ser útil muy rapidamente a medida que crece $k$. Por el contrario, tanto la interpolación bilineal como la
interpolación por splines deberian funcionar mejor para valores de $k$ grandes, en el caso de esta \'ultima, siendo la que menos ruido introduzca
(por la necesidad de que el polinomio sea derivable en cada intersección, suavizando el cambio de uno a otro).
