A continuación presentaremos las técnicas mencionadas con su respectivo análisis:

\subsection{Vecinos}
El primer método que analizaremos será el de vecinos. Este consiste en rellenar los valores de cada una de las columnas
nuevas de la imagen replicando su valor más próximo.

% ACA ARRANCA LOS DATOS GROSOS DE VECINOS

\subsection{Interpolación bilineal}
El segundo método que analizaremos será el de interpolación bilineal. En este caso, la idea consiste en generar un polinomio entre
dos puntos consecutivos de la imagen para, por medio de este, calcular los valores necesarios para la extensión.
Primero realizaremos el cálculo por filas y una vez calculados estos valores, repetiremos el mismo procedimiento por columnas.

% DATOS GROSOS DE INTERPOLACION

\subsection{Interpolación por Splines}
Por último nos centraremos en la interpolación por Splines. Este método, similar al anterior, requiere el cálculo de Splines
(al igual que antes, por filas y luego por columnas) para obtener los valores de los casilleros a extender.
Además, en este caso, dado que la interpolación por medio de Splines trata de generar polinomios para un segmento
específico de la imagen, también analizaremos una versión en la cual el Spline calculado solo incluye los valores de un recuadro de tamaño menor,
dejando de lado los valores de la imgen externos (que estan influyendo sobre un polinomio que intenta interpolar valores posiblemente lejos de los suyos).

% DATOS DE SPLINES
