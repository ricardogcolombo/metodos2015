\subsection{Compilación y formato de ejecución del programa}

\subsubsection{Compilación}

Como se encuentra mencionado en el archivo $README.txt$, en la carpeta $src$ de la carpeta raíz del trabajo se encuentra un $Makefile$. Así es que ejecutando el comando

$\texttt  make $

se genera el ejecutable $tp$.

\subsubsection{Formato de ejecución}

El comando para correr el experimento para una instancia es: 

$\texttt ./tp archivoEntrada metodo k $

Donde:

\begin{itemize}
  \item $archivoEntrada$ es la imagen original
  \item $metodo$ es un entero entre 0 y 2 donde 
	
  \begin{itemize}
    \item 0 es el metodo de vecinos
    \item 1 es el metodo Bilineal
    \item 2 es el metodo de Splines
  \end{itemize}

  \item $k$ es un entero que indica cuanto zoom se hará en la foto
\end{itemize}
