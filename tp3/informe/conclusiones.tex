Como se menciono con anterioridad, nuestra intuicion relacionaba fuertemente a los metodos en cuanto a calidad/desempe\~no.
Como se pudo ver a lo largo del analisis realizado, el metodo de splines nunca logro sacar una diferencia significativa respecto al metodo de interpolacion bilineal. Tambien se puede apreciar como este ultimo tiene un mejor desempe\~no temporal en todos los casos.
Esto coloca a la interpolacion bilineal como la mejor opcion en cuanto a tiempo y calidad, dado que la ganancia por splines es minima respecto al tiempo extra. En un momento (K = 2), nos llamo poderosamente la atencion que el metodo bilineal haya obtenido un mejor resultado que el metodo de splines, pero teniendo en cuenta como se terminaron comportando ambos metodos a lo largo de todo el analisis, ahora ya no parece un resultado tan anomalo. Sin embargo, no logramos llegar a una conclusion de porque la diferencia es tan significativa.

En cuanto al metodo de los vecinos, como bien dijimos al principio, se comporta relativamente bien para valores de $k$ minimos, pero enseguida que este crece, el metodo pierde fiabilidad.