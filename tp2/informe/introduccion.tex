El objetivo de este trabajo será reconocer imagenes que contienen digitos atravez de la utilizacion de tecnicas simples de machine learning.
\\
La metodologia consistira en la siguiente. Tendremos una base de datos con imagenes ya etiquetadas y una base de datos con imagenes sin etiquetas.
\\
El objetivo se sentrará en que el sistema pueda utilizar la base de datos para poder etiquetar de manera correcta la base de datos sin etiquetar.
\\
Para ello utilizaremos primero el metodo mas intuitivo posible. Esto es para cada imagen a etiquetar, buscamos la que mas se le parezca en la base de datos etiquetada, y decimos que esa es su etiqueta. Por supuesto, todavia queda determiar que es que dos imagenes se 'parezcan'. Eso será descripto en mayor profundidad en el siguiente apartado.
\\
Como segunda idea intuitiva podemos pensar que tal vez tenemos mucha mala suerte y la imagen mas parecida es la de una etiqueta erronea, luego la manera de resolver esto es, en vez de tomar un solo vecino, tomar $k$ vecinos y de esos $k$ vecinos mas cercanos ver que etiqueta es la que mas veces se repite. Estas ideas basicas, es la de $KNN$ que será explicada en mas detalle en el proximo apartado de este trabajo.
\\
Luego,a esta idea, intentaremos aplicarle una mejora sustancial utiliando un metodo probabilistico conocido como $PCA$ que consistira en aplicarle una transformacion a la imagen de tal manera de solo quedarnos con aquello de mayor variabilidad y desechar aquello que pueda estar introduciendo ruido.