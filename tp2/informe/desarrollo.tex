\subsection{Algortimo de kNN}
Como primera aproximaci\'on para la resoluci\'on del problema de OCR, implementamos el algoritmo de $K vecinos mas cercanos$ (o $kNN$ por sus siglas en ingles). Este algoritmo consiste basicamente en la idea de que entradas parecidas, a partir de una metrica definida en la implementacion (que para este caso podria variar desde, por ejemplo, desde la cantidad de puntos arriba de cierto valor para contabilizar la cantidad de valores $negros$ y $blancos$ hasta la norma de cada vector digito) presentaran caracteristicas definidas y al ser ubicadas sobre un plano se agruparan de acuerdo a estas. Luego, para clasificar un nuevo objeto, basta con ubicarlo dentro de este plano y promediar la etiqueta de los $k$ vecinos mas cercanos para obtener una clasificaci\'on. Sin embargo, y mas alla de las mejoras que puedan realizarse sobre los datos en crudo, este algoritmo es muy sensible a la variabilidad de los datos. Un conjunto de datos con un poco de dispersion entre las distintas clases de clasificacion, hace empeorar rapidamente los resultados.
Analizaremos en la segunda etapa del trabajo practico, una forma de solucionar este incoveniente mediante el an\'alisis de componentes principales.

//ACA UN PSEUDOCODIGO?

\subsection{Optimizacion mediante An\'alisis de componentes principales}
En esta segunda parte, utilizaremos una tecnica conocida como $analisis de componentes principales$ como una forma de optimizar los resultados de la primera etapa. El analisis de componentes principales (o PCA) consiste basicamente en conseguir una descomposicion de los datos en sus matrices ortogonales de valores principales para obtener una transformacion lineal que resuma la informaci\'on mas relevante de cada imagen, descartando aquellos valores que no aportan datos y resultan redundantes.
Para realizar este procedimiento tomamos la matriz de covarianza como una forma de expresar la relacion de dependecia intrinseca entre cada variable. A partir de esta informaci\'on y mediante el metodo de la potencia, obtenemos un vector $P$ que, al multiplicarlo por nuestros valores originales, realiza el cambio de coordenadas minimizando la covarianza.

//ACA UN PSEUDOCODIGO?

\subsection{Cross-validation}
