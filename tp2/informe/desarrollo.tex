\subsection{Algortimo de kNN}
Como primera aproximación para la resolución del problema de OCR, implementamos el algoritmo de $K$-vecinos más cercanos (o $kNN$ por sus siglas en inglés). Este método de clasificación consiste básicamente en, dado un dato del que no conocemos a que clase pertenece, buscar entre las imágenes del dataset etiquetado las $k$ más parecidos, llamados también como sus "vecinos" (habiendo que definir que es ser "parecido"), y luego de estos $k$ vecinos, determinar cual es la moda. 
\\
\subsubsection{Similitud entre imágenes}
Para este trabajo en particular, tomamos las imágenes como vectores numéricos y definimos que dos imágenes son "parecidas" si la norma dos entre ellas es pequeña. Luego la idea del $knn$ será tomar todas las imágenes etiquetadas, compararlas contra la nueva imagen a reconocer, ver cuales son las $k$ imágenes cuya norma es la menor posible y, entre esos k vecinos, ver a que clase pertenecen. La etiqueta para esta imagen vendrá dada por la moda. 

Para los siguientes pseudocódigos será necesario asumir que todas las estructuras utilizadas almacenan datos enteros a menos que se indique lo contrario, esto se indica agregando entre paréntesis el tipo de dato que almacena.
\\
\begin{algorithm}
\begin{algorithmic}[1]\parskip=1mm
\caption{Vector KNN(matriz etiquetados, matriz sinEtiquietar,int cantidadVecinos)}
\STATE{vector etiquetas =  vector(cant$\_$filas(sinEtiquetar))}
\FOR{1 \TO cant$\_$filas(sinEtiquetar)}
	\STATE{$etiquetas_i$ = encontrarEtiquetas(etiquetados,sinEtiquetar$_{i}$,cantidadVecinos)}
\ENDFOR
\RETURN{etiquetas}
\end{algorithmic}
\end{algorithm}

\begin{algorithm}
\begin{algorithmic}[1]\parskip=1mm
\caption{int encontrarEtiquetas(matriz etiquetados, vector incognito,int cantidadVecinos)}
\STATE{colaPrioridad(norma,etiqueta,vectorResultado)  resultados}
\FOR{1 \TO size(incognito)}
	\STATE{resParcial $=$ restaVectores($etiquetados_i$,incognita)\\}
	\STATE{colaPrioridad.push((norma(resParcial),etiqueta($etiquetados_i$))}
\ENDFOR
\STATE{vector numeros = vector(10)}
\WHILE{cantidadVecinos$>$0 $\&$  noesVacia(resultados)}
	\STATE{int elemento $= $primero(resultados.etiqueta)  \\}
	\STATE{$numeros_{elemento}$ ++ }
\ENDWHILE\\
\RETURN{maximo(numeros)}
\end{algorithmic}
\end{algorithm}

Una supocición preliminar es que a mayor cantidad de vecinos (o sea, $k$) menor va a ser la cantidad de aciertos, ya que se empiezan a mirar los elementos de menor prioridad de la cola, eso significa, que se cuentan primero las imágenes que más difieren y eso puede hacer que las chances de acertar el dígito correcto disminuyan. Ahondaremos mas en detalle en la siguiente sección, cuando pongamos a prueba cual es la mejor cantidad de vecinos para este algoritmo.
\\
Al comienzo del desarrollo de los experimentos pensamos en diferentes maneras de mejorar el procesamiento de las imágenes, ya sea pasandolas a blanco y negro para no tener que lidiar con escala de grises o recortar los bordes de las imágenes, ya que en ellos no hay demasiada información útil (en todas las imágenes vale 0).
\\
Sin embargo, y mas allá de las mejoras que puedan realizarse sobre los datos en crudo, este algoritmo es muy sensible a la variabilidad de los datos. Un conjunto de datos con un cierto grado de dispersión entre las distintas clases de clasificación hace empeorar rápidamente los resultados.
\\
En el siguiente apartado pasaremos a describir una metodología más sofisticada para resolver este problema que mejora tanto los tiempos de ejecución como la tasa de reconocimiento con respecto al método descripto anteriormente.

\subsection{Optimización mediante Análisis de componentes principales}
El Análisis de Componentes Principales o $PCA$ es un procedimiento probabilístico que utiliza una transformación ortogonal para convertir un conjunto de variables, posiblemente correlacionadas, en un conjunto de variables linealmente independientes llamadas componentes principales.
\\
Esta transformación está definida de tal manera que la primera componente principal tenga la varianza más grande posible, la segunda componente tenga la segunda varianza más grande posible y así sucesivamente hasta encontrarse con la componente de menor varianza en la última posición.
\\
De esta manera será fácil quedarnos con los $\lambda$ componentes principales que concentren la mayor varianza y quitar el resto. En la sección de experimentación, uno de los objetivos principales será buscar cual es el $\lambda$ que concentra la mayor varianza de manera tal de optimizar el número de predicciones. 
\\
A fines prácticos, lo que hacemos es, a partir de nuestra base de datos de elementos etiquetados, construir la matriz de covarianza $M$ de tal manera que en la coordenada $M_ij$ obtenga el valor de la covarianza del pixel $i$ contra el pixel $j$.
\\
Luego, utilizando el método de la potencia, procedemos a calcular los primeros $\lambda$ autovectores de esta matriz. Una vez obtenidos los autovectores multiplicamos cada elemento por los $\lambda$ autovectores y asi obtenemos un nuevo set de datos.
\\
Sobre este set de datos, ahora aplicamos el algoritmo $KNN$ nuevamente y lo que esperamos ver es un mayor número de aciertos, ya que hemos quitado ruido del set de datos, sumado a mejores tiempos de ejecución, ya que hemos reducido la dimensionalidad del problema.
\subsection{Cross-validation}
Para medir la precisión de nuestros resultados utilizamos la metodología de cross-validation. Esta consiste en tomar nuestra base de datos de entrenamiento y dividirla en $k$ bloques. En una primera iteracion se toma un bloque para testear y los bloques restantes para entrenar a nuestro modelo, observando los resultados obtenidos. En la siguiente, se toma el segundo bloque para testear y los restantes como dataset de entrenamiento. La metodología se repite $k$ veces hasta iterar todo el conjunto de datos. Finalmente, se realiza la media aritmetica de los resultados de cada iteracion para obtener un unico resultado de error y poder evaluar la performance del metodo de entrenamiento.


Esta técnica, que es una mejora de la técnica de holdout donde simplemente se divide el set de datos en dos conjuntos (uno para entrenamiento y otro para testing), trata de garantizar que los resultados obtenidos sean independientes de la partición de datos contra la que se esta evaluando porque ofrece el beneficio de que los parámetros del método de predicción no pueden ser ajustados exhaustivamente a casos particulares. Es por esto que se utiliza principalmente en situaciones de predicción, dado que intenta evitar que el aprendizaje se realice sobre un cuerpo de datos especifico y busca obtener respuestas mas generales.


La única desventaja que presenta es la necesidad esperable de correr los algoritmos en varias iteraciones, situacion que puede tener un peso significativo si el método de predicción tiene un costo computacional muy alto durante el entrenamiento. 

\subsection{Algoritmo PCA}

\newpage
\begin{algorithm}
\begin{algorithmic}[1]\parskip=1mm
\caption{void PCA(matriz etiquetados, matriz sinetiquetar,int cantidadAutovectores)}
\STATE{matriz covarianza $=$  obtenerCovarianza(etiquetados)}
\STATE{vector(vector) autovectores}  
\FOR{1 \TO cantidadAutovectores}
	\STATE{vector autovector$ = $metodoDeLasPotencias(covarianza)}
	\STATE{agregar(autovectores,autovector)}
	\STATE{double lamda = encontrarAutovalor(auovector,covarianza)}
	\STATE{multiplicarXEscalar(auovector,lamda)}
	\STATE{restaMatrizVector(covarianza,auovector,lamda)}
\ENDFOR\\
\end{algorithmic}
\end{algorithm}

\begin{algorithm}
\begin{algorithmic}[1]\parskip=1mm
\caption{matriz obtenerCovarianza(matriz entrada,vector medias)}
\STATE{matriz covarianza, vector nuevo}
\FOR{i=1 \TO size(medias)}
	\FOR{j=1 \TO cant\_filas(entrada)}
		\STATE{\quad $nuevoVector_j= entrada_{(j,i)}- medias_i$ }
	\ENDFOR
	\STATE{agregar(covarianza,nuevoVector)}
\ENDFOR
\FOR{i=1 \TO cant\_filas(entrada)}
	\FOR{k=1 \TO cant\_filas(entrada)}
		\STATE{$covarianza_i$ $=$ multiplicarVectorEscalar($covarianza_k$, cantidad\_filas(entrada)}
	\ENDFOR
\ENDFOR
\RETURN{covarianza}
\end{algorithmic}
\end{algorithm}


\begin{algorithm}
\begin{algorithmic}[1]\parskip=1mm
\caption{Vector metodoDeLasPotencias(matriz covarianza,cantIteraciones)}
\STATE{vector vectorInicial$=$  vector(cant\_filas(covarianza))}
\FOR{1 \TO cantIteraciones}
	\STATE{vector nuevo $=$ multiplicar(covarianza,vectorInicial)}
	\STATE{multiplicarEscalar(nuevo,1/norma(nuevo))}
	\STATE{vectorInicial $=$ nuevo}
\ENDFOR
\RETURN{vectorInicial}
\end{algorithmic}
\end{algorithm}

\begin{algorithm}
\begin{algorithmic}[1]\parskip=1mm
\caption{Vector medias(matriz entrada)}
\STATE{vector medias$ =  $vector(cant\_columnas(entrada))}
\FOR{i$=$1 \TO cant\_columnas(entrada)}
	\STATE{suma $=$ 0}
	\FOR{j$=$1 \TO cant\_columnas(entrada)}
		\STATE{\quad suma $+=$ $entrada_{i,j}$}
	\ENDFOR
	\STATE{$medias_i =$ suma/cant\_filas(entrada)}
\ENDFOR\\
\RETURN{vectorInicial}
\end{algorithmic}
\end{algorithm}

