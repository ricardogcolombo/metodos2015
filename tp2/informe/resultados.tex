\subsection{Resultados del testeo}
Del apartado de experimentación podemos deducir que el mejor método de predicción es el método de $knn$ que, con $k=3$, obtiene alrededor de un $96 \%$ de aciertos. Sin embargo esto viene asociado con tiempos de ejecución muy altos y que podrían no resultar apropiados. Recordemos que en el apartado anterior determinamos que para clasificar $4200$ imágenes obtuvimos tiempos cercanos a los $4.35$ minutos en máquinas modernas.
\\
También vimos que el PCA, si bien no llega a tasas tan altas de predicción como el $KNN$, obtiene resultados aceptables para $k = 5$ y $\alpha = 14$ encontrados, que rondan entre un $92 \%$ y $90 \%$. La ventaja de este método es que sus tiempos de ejecución son mucho menores que los del $KNN$, clasificar $4200$ imágenes tardaba $0.36$ minutos, o sea un $8.4 \%$ de lo que tardaba la metodología de $KNN$!. 
\\
Luego a la hora de elegir alguna de estas dos metodologías de resolución uno debe sopesar que es lo que más le interesa, resultados rapidos pero con una menor tasa de aciertos o resultados más precisos pero que requieren de tiempos elevados de procesamiento.
