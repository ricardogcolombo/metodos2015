\subsection{Resultados del testeo}
Del apartado de experimentación podemos deducir que el mejor metodo de prediccion es el metodo de $knn$, que, con un $k=3$, obtiene al rededor de un $96 \%$ de aciertos. Sin embargo esto viene asociado con tiempos de ejecución increiblemente altos y que podrían no resultar apropiados. Recordemos que en el apartado anterior determinamso que para clasificar $4200$ imagenes obtuvimos tiempos cercanos a los $4.35$ minutos.
\\
Tambien vimos que el PCA, si bien no llega a tasas tan altas de prediccion como el $KNN$, obtiene resultados aceptables para el mejor $k$ y $\alpha$ encontrados, que rondan entre un $92 \%$ y $90 \%$. La ventaja de este metodo es que sus tiempos de ejecución son mucho menores que los del $KNN$, clasificar $4200$ imagenes tardaba $0.36$ minutos, osea un $8.4 \%$ de lo que tardaba la metodologia de $KNN$!. 
\\
Luego a la hora de elejir alguna de estas dos metodologías de resolución uno debe sopesar que es lo que mas le interesa, resultados rapidos pero con una menor tasa de aciertos o resultados mas precisos pero que requieren de tiempos elevados de procesamiento.
