Entre las conclusiones que podemos destacar luego del análisis realizado se encuentra que:

\begin{enumerate}
 \item El algoritmo KNN presenta una efectividad más que aceptable, entendiendo que es una técnica que cuenta con varios años de antiguedad.
 \item Consideramos importante destacar que esto depende en gran medidad de la variación de los datos a analizar. En aquellos conjuntos donde la varianza es elevada y los datos se encuentran muy dispersos, promediar el resultado en base a sus vecinos más cercanos puede no resultar la mejor técnica a implementar.
 \item Teniendo en cuenta esto, la relación costo-beneficio de la implementación y ejecución previa de una optimización como la del algoritmo de $PCA$, resulta mínima. En todos los casos los resultados mejoraron, dado que concentrar los factores relevantes en componentes especificas del set de datos favorece la vecindad de la que el algoritmo de $KNN$ hace uso.
 \item Dada la característica principal del algoritmo de $PCA$ (ordenar las componentes principales en base a su relevancia), se permite ajustar la cantidad de datos a considerar, dando lugar a una mejora no solo en los resultados, sino también a la performance y al uso de memoria. Como vimos durante nuestro análisis, la cantidad óptima está bastante por debajo del máximo y no tienen ningún beneficio considerar una mayor cantidad de estas.
\end{enumerate}
