El análisis realizado nos lleva a sacar una serie de conclusiones en base a lo experimentado.
\\
El algoritmo KNN presenta una efectividad más que aceptable, entendiendo que es una técnica que cuenta con varios años de antiguedad.
\\
Consideramos importante destacar que esto depende en gran medidad de la variación de los datos a analizar. En aquellos conjuntos donde la varianza es elevada y los datos se encuentran muy dispersos, promediar el resultado en base a sus vecinos más cercanos puede no resultar la mejor técnica a implementar.
\\
Teniendo en cuenta esto, la relación costo-beneficio de la implementación y ejecución previa de una optimización como la del algoritmo de $PCA$, resulta mínima. En todos los casos los resultados mejoraron, dado que concentrar los factores relevantes en componentes especificas del set de datos favorece la vecindad de la que el algoritmo de $KNN$ hace uso.
\\
Dada la característica principal del algoritmo de $PCA$ (ordenar las componentes principales en base a su relevancia), se permite ajustar la cantidad de datos a considerar, dando lugar a una mejora no solo en los resultados, sino también a la performance y al uso de memoria. Como vimos durante nuestro análisis, la cantidad óptima está bastante por debajo del máximo y no tienen ningún beneficio considerar una mayor cantidad de estas.
\\
Si bien estos algoritmos nos muestran resultados interesantes, los tiempos de ejecución utilizando hardware moderno son altos. Este se puede mejorar utilizando técnicas de paralelización utilizando multiples cores o también utilizando instrucciones SIMD (si el procesador lo soporta), sin embargo estos tiempos no son aceptables para aplicaciones que requieren realizar OCR en \"tiempo real\".
\\
Como se menciona al comienzo del trabajo, el preprocesamiento de las imágenes es otro factor que puede mejorar la eficiencia algorítmica. Así como $PCA$ quita ruido del dataset, es posible homogeneizar las imágenes por separado aplicando otros filtros.
\\
Si bien el propósito del trabajo busca encontrar dígitos en imágenes este mecanismo se puede utilizar de un modo muy parecido para encontrar otras características tanto en imágenes como en audios y así etiquetar según clases que no tienen que ver necesariamente con la extracción de dígitos. 
