
Entre las conclusiones que podemos destacar luego del analisis realiza se encuetra que:
\begin{enumerate}
 \item El algoritmo KNN presenta una efectividad mas que aceptable, entendiendo que es una tecnica que cuenta con varios años de antiguedad.
 \item Consideramos importante destacar que esto depende en gran medidad de la variacion de los datos a analizar. En aquellos conjuntos donde la varianza es elevada y los datos se encuentran muy dispersos, promediar el resultado en base a sus vecinos mas cercanos, puede no resultar la mejor tecnica a implementar.
 \item Teniendo en cuenta esto, la relacion costo-beneficio de la implementacion y ejecucion previa de una optimizacion como la del algoritmo de PCA, es minima. En todos los casos los resultados mejoraron, dado que contentrar toda la importancia en componentes especificas del set de datos, favorece la vecindad de la que el algoritmo de KNN hace uso.
 \item Dada la caracteristica principal del algoritmo de PCA (ordenar las componentes principales en base a su relevancia), se permite ajustar la cantidad de datos a considerar, dando lugar a una mejora no solo en los resultados, sino tambien a la performance y al uso de memoria. Como vimos durante nuestro analisis, la cantidad optima esta bastante por debajo del maximo y no tienen ningun beneficio considerar una mayor cantidad de estas.
\end{enumerate}