\documentclass[a4,11pt]{article}

\parindent=10pt
\parskip=6pt
%\usepackage[width=15.5cm, left=2.5cm, top=2cm, height= 24.5cm]{geometry}

\usepackage[paper=a4paper, left=2cm, right=2cm, bottom=2.5cm,top=2.5cm]{geometry}

% Paquetes de nacionalización. No olvidar para poder poner tildes!
\usepackage[spanish]{babel}
\usepackage[utf8]{inputenc}

% Paquetes para graficos
\usepackage{subfig}
% \usepackage{graphicx} %% La caratula lo incluye

% Paquetes para matematica
\usepackage{amsmath}
\usepackage{amsfonts}
\usepackage{amssymb}

% Paquetes para pseudo
\usepackage{algorithm}
\usepackage{algorithmic}

% Caratula (Recordar logo_uba.jpg y logo_dc.jpg)
\usepackage{caratula}

% Paquetes para tablas
\usepackage[table]{xcolor}

% Se pueden sacar?
\usepackage{url}
\usepackage{float}
\usepackage{afterpage}
\usepackage{tabularx}

% Color de links
\usepackage{hyperref}
\hypersetup{
    colorlinks,
    citecolor=black,
    filecolor=black,
    linkcolor=black,
    urlcolor=black
}

\begin{document}


\materia{Metodos numericos}
\submateria{Primer Cuatrimestre de 2015}
\titulo{Trabajo Pr\'actico 1}
\subtitulo{“Si nos organizamos aprobamos todos...”}
\integrante{Gastón Zanitti}{058/10}{gzanitti@gmail.com}
\integrante{Ricardo Colombo}{156/08}{ricardogcolombo@gmail.com}
\integrante{Dan Zajdband}{144/10}{Dan.zajdband@gmail.com}
\integrante{Franco Negri}{893/13}{franconegri200@gmail.com}


\maketitle
\pagebreak
  
\tableofcontents

\pagebreak

\section{Introduccion}

El objetivo de este trabajo será reconocer imagenes que contienen digitos atravez de la utilizacion de tecnicas simples de machine learning.
\\
La metodologia consistira en la siguiente. Tendremos una base de datos con imagenes ya etiquetadas y una base de datos con imagenes sin etiquetas.
\\
El objetivo se sentrará en que el sistema pueda utilizar la base de datos para poder etiquetar de manera correcta la base de datos sin etiquetar.
\\
Para ello utilizaremos primero el metodo mas intuitivo posible. Esto es para cada imagen a etiquetar, buscamos la que mas se le parezca en la base de datos etiquetada, y decimos que esa es su etiqueta. Por supuesto, todavia queda determiar que es que dos imagenes se 'parezcan'. Eso será descripto en mayor profundidad en el siguiente apartado.
\\
Como segunda idea intuitiva podemos pensar que tal vez tenemos mucha mala suerte y la imagen mas parecida es la de una etiqueta erronea, luego la manera de resolver esto es, en vez de tomar un solo vecino, tomar $k$ vecinos y de esos $k$ vecinos mas cercanos ver que etiqueta es la que mas veces se repite. Estas ideas basicas, es la de $KNN$ que será explicada en mas detalle en el proximo apartado de este trabajo.
\\
Luego,a esta idea, intentaremos aplicarle una mejora sustancial utiliando un metodo probabilistico conocido como $PCA$ que consistira en aplicarle una transformacion a la imagen de tal manera de solo quedarnos con aquello de mayor variabilidad y desechar aquello que pueda estar introduciendo ruido.

\pagebreak
\section{Desarrollo}

A continuación presentaremos el desarrollo de los experimentos que exploran las técnicas mencionadas en la intriducción con su respectivo análisis:

\subsection{Vecinos mas cercanos}
El primer método que analizaremos será el de vecinos. Este consiste en rellenar los valores de cada una de las columnas
nuevas de la imagen replicando su valor más próximo. La principal ventaja de este método es la simpleza de su implementación, que consta únicamente de dos bucles para iterar la matriz original. Dicha característica también le provee de una eficiencia del orden de O($nxm$) siendo $n$ el alto y m ancho de la imagen destino.

\begin{algorithm}[H]
\begin{algorithmic}[1]\parskip=1mm
\caption{void vecinos(Matriz *image, Matriz *imageRes , int k)}
\FOR{0 \TO imageRes$\rightarrow$rows - 1}
    \FOR{0 \TO imageRes$\rightarrow$cols - 1}
        \STATE{imageRes$\rightarrow$at(i, j) $=$ image$\rightarrow$at(round(i/(k+1)), round(j/(k+1)))\\}
    \ENDFOR
\ENDFOR
\end{algorithmic}
\end{algorithm}

Como se mencionó anteriormente, a pesar su alta eficiencia temporal en comparación a los demás métodos, nuestra intuición nos dice que este va ser el que presente una mayor cantidad de ruido, debido a que simplemente se están replicando los píxeles de las imágenes. Como resultado, teniendo en cuenta que lo único que se logra es "añadir grosor" a los píxeles, deberían conseguirse imágenes de mayor tamaño pero con una ganancia igual de rápida en los valores de ruido a medida que los valores de $k$ aumentan.

\subsection{Interpolación bilineal}
El segundo método que analizaremos será el de interpolación bilineal. En este caso, la idea consiste en generar un polinomio entre
dos puntos consecutivos de la imagen para, por medio de este, calcular los valores necesarios para la extensión. \\
Primero realizaremos el cálculo por filas y una vez calculados estos valores, repetiremos el mismo procedimiento por columnas.
Sean entonces $Q_{11}$, $Q_{12}$, $Q_{21}$, $Q_{22}$ los cuatro puntos de la imagen original sobre los que queremos interpolar, el objetivo es conseguir un polinomio P que valga lo mismo en cada uno de estos puntos y aproxime los nuevos valores intermedios. Usaremos entonces para esto el polinomio interpolador de Lagrange.

Interpolando entonces en el eje X obtenemos la siguiente fórmula:

\begin{center}
\includegraphics[scale=0.75]{imagenes/bilinealX.png}\\
\end{center}


Ahora, realizando el mismo procedimiento pero en el eje Y, obtenemos lo siguiente:

\begin{center}
\includegraphics[scale=0.75]{imagenes/bilinealY.png}\\
\end{center}

Si notamos, los puntos que acompañan a las bases polinómicas de Lagrange son los mismos que calculamos sobre el eje X, por lo que podemos realizar el remplazo para llegar a una fórmula cerrada:

\begin{center}
\includegraphics[scale=0.75]{imagenes/bilinealXY.png}\\
\end{center}

Distribuyendo los valores dentro de los paréntesis, obtenemos la ecuación final

\begin{center}
\includegraphics[scale=0.75]{imagenes/bilinealFinal.png}\\
\end{center}

Notar que tanto los valores del X e Y de los puntos con los que generamos la formula, como el valor en el punto son constantes que no varian mientras mantengamos los 4 pixeles escogidos con lo cual podriamos reutilizar esto para los demas calculos y ademas la distancia de los puntos $x_1$ y $x_2$ es 1 como el de los valores de $y_1$ e $y_2$ por lo que el denominador en la formula se puede quitar, quedandonos una formula de una recta. 

\begin{center}
$f(x)= Q_{11}.valor* (Q_{22}.x-x)*(Q_{22}.y-y) + Q_{21}.val * (x-Q_{11}.x)*(Q_{22}.y-y)+ Q_{12}.valor* (Q_{22}.x-x)*(y-Q_{11}.y) + Q_{22}.val * (x-Q_{11}.x)*(y-Q_{11}.y) $
\end{center}

Por tanto si se hace primero sobre el eje X la ecuacion de la recta y luego sobre el Y, como a la inversa. Ahora, gracias a esta formula, podemos conseguir los valores de las posiciones $(x,y)$ que agregamos a nuestra imagen para realizar el zoom.\\
Para facilitar la lectura y escritura del ejercicio vamos a definir una estructura que se llama punto. Dentro de la misma vamos a tener 3 valores, el primero es el valor en x, el segundo en y y el tercero un valor que sera de la imagen original. 
En el siguiente codigo las variables $q11$, $q12$, $q21$, $q22$ son del tipo descripto anteriormente y se utilizan para definir los 4 puntos en los cuales se va a realizar la formula de la recta evaluada en el punto, la matriz A representa la imagen original y la matriz Res representa la imagen extendida.
Veamos el siguiente ejemplo para aclarar quienes son los pixeles que utilizamos, supongamos los primeros 4 pixeles de la siguiente forma de una imagen.
\\
Luego utilizando k = 2 agrandamos la imagen dejando en el medio 2 pixeles entre cada pixel de la imagen original.
\begin{table}[H]
\begin{center}
\caption{pixeles de imagen original}
\begin{tabular}{|l|l|}
\hline
q11 & q12\\
\hline
q21 & q22\\
\hline
\end{tabular}
\end{center}
\end{table} 

\begin{table}[H]
\begin{center}
\caption{Pixeles de imagen aumentada}
\begin{tabular}{|l|l|l|l|}
\hline
q11 & a & b & q12\\
\hline
c & d & e & f\\
\hline
g & h  & i & j \\
\hline
q21 & k &  l & q22\\
\hline
\end{tabular}
\end{center}
\end{table}
Como los valores de q11,q12,q21 y q22 son los valores de la imagen original y podemos realizar la formula con los mismos.

\begin{algorithm}
\begin{algorithmic}[H]\parskip=1mm
\caption{void bilineal(matriz A, vector Res,int k)}
  \STATE{Para i$=0...CantFilas-1$}
  \STATE{\quad Para j$=0...CantColumnas-1$}
  \STATE{\quad \quad q11 = $<0,0, A_{i,j}>$}
  \STATE{\quad \quad q12 = $<0,k+1, A_{i,j+1}>$}
  \STATE{\quad \quad q21 = $<k+1,0, A_{i+1,j}>$}
  \STATE{\quad \quad q22 = $<k+1,k+1, A_{i+1,j+1}>$}
  \STATE{\quad \quad Para x=$0...k+1$}
  \STATE{\quad \quad \quad Para y=$0...k+1$}
  \STATE{\quad \quad \quad \quad valorRes $=$ poliniomioInterpolador(q11,q12,q21,q22,x,y) }
  \STATE{\quad \quad \quad \quad $Res_{i*(k+1)+x,j*(k+1)+y} =$ valorRes}
\end{algorithmic}
\end{algorithm}
Donde $polinomio interpolador$ es la función que se encarga de generar el polinomio interpolador(que en este caso es una recta) en el punto, de la siguiente manera:
\begin{algorithm}
\begin{algorithmic}[H]\parskip=1mm
\caption{void polinomioInterpolador(punto q11,punto q12, punto q21, punto q22, int x, int y)}
  \STATE{denominador = 1/ ((q22.x-q11.x)* (q22.y-q11.y)) }
  \STATE{numerador1= q11.valor* (q22.x-res.x)*(q22.y-res.y) + q21.val * (res.x-q11.x)*(q22.y-res.y)}
  \STATE{numerador2= q12.valor* (q22.x-res.x)*(res.y-q11.y) + q22.val * (res.x-q11.x)*(res.y-q11.y)}
  \STATE{retorno ((numerador1+numerador2)*denominador)}
\end{algorithmic}
\end{algorithm}
Para el caso del polinomio interpolador se agregaron 2 lineas al final de la rutina, las cuales aplican saturación en caso de ser necesario cuando luego de realizar todas las cuentas el valor que nos queda es mayor a 255, fijandolo en 255, y cuando el valor es menor a 0, fijandolo en 0.
\\
En este método, comparado con el anterior que solo replicaba píxeles vecinos, se está calculando un polinomio para tratar de introducir cierto nivel de suavidad entre los puntos de la imagen original a medida que se recorren los píxeles. Un dato importante a tener en cuenta es que dado que el grado del polinomio aumenta a medida que la cantidad de puntos a interpolar es mayor, decidimos que esta se realice entre solo dos puntos de la imagen original, para ofrecer un mejor desempe\~no entre puntos, haciendo que el polinomio calculado sea mas operativo y evitando que este oscile demasiado (situación conocida como Fenómeno de Runge). 
\\
\\
La complejidad de este algoritmo no es alta. Se recorren n filas y para cada una de ellas m columnas para recorrer toda la matriz, luego se fijan 4 puntos a procesar (que eso tiene complejidad O(1) para acceder a la posición de la matriz) y para cada conjunto de puntos, se realizan 2 ciclos de complejidad O(k), cada uno para recorrer los píxeles cercanos. Luego se genera el polinomio interpolador, el denominador se consigue a través de 2 restas y 1 multiplicación, y el numerador tiene un costo de 12 multiplicaciones y 8 restas. Por ultimo, se multiplican el numerador con el denominador, por lo tanto armar el polinomio interpolador cuesta 10 restas y 14 multiplicaciones.
Si consideramos que las restas y multiplicaciones no son muy costosas y que crear el polinomio cuesta O(1) (a muy grandes rasgos), la complejidad del algoritmo bilineal seria O(n*m*k\^{2}) con n cantidad de filas, m cantidad de columnas y k cantidad de filas/columnas agregadas entre 2 filas/columnas, que por lo general en la práctica, el k es muchisimo más chico que n y m, el tamaño de la imágen.


\subsection{Interpolación por Splines}
Por último nos centraremos en la interpolación por splines. Este método, similar al anterior, requiere el cálculo de Splines (al igual que antes, por filas y luego por columnas) para obtener los valores de los casilleros a extender.

Decidimos para este caso utilizar splines naturales. Recordemos que la condición de suavidad de los splines naturales es que $S''(a) = 0$ y $S''(b) = 0$. Esto determina que la interpolación en los bordes va a ser suave.

Además, en este caso, dado que la interpolación por medio de splines trata de generar polinomios para un segmento
específico de la imagen, realizaremos un análisis sobre el algoritmo de interpolación por splines para tratar de obtener el tamaño de ventana más óptimo. Dado que este algoritmo solo incluye los valores de un recuadro de tamaño específico, creemos que agregar más valores puntos a considerar por el spline, no presentará beneficio alguno porque se estarían dejando de lado los valores de la imagen externos al punto a calcular (que estarían influyendo sobre un polinomio que intenta interpolar valores posiblemente lejos de los suyos).

Sea $S_{j}(x) = a_{j} + b_{j}(x - x_{j}) + c_{j}(x - x_{j})^2 + d_{j}(x - x_{j})^3$ nuestro polinomio interpolador para cada $j$ intervalo entre $0$ y $n-1$, necesitamos entonces resolver los coeficientes. Utilizamos la construcción de Splines despejando estos últimos en función de $c_{j}$ para formar el siguiente sistema de ecuaciones $Ax = b$:

\begin{center}
\includegraphics[scale=0.50]{imagenes/A.png}
\includegraphics[scale=0.50]{imagenes/x.png}
\includegraphics[scale=0.50]{imagenes/b.png}
\end{center}

donde $h = x_{j+1} - x_{j}$. Para nuestra aplicación $h = k$ y se mantiene fijo, ya que la distancia entre los puntos de la nueva imagen es igual a $k$. Una vez resuelto este sistema y con los valores de $c_{0},..., c_{n}$ ya calculados, podemos despejar los coeficientes que necesitabamos en base a las siguientes ecuaciones (que se derivan de las condiciones del mismo spline):

$a_j$ es el valor del pixel en la posición $j$ de la imagen original en la fila o columna iterada por el spline \\ \\
$b_{j} = \frac{1}{k}(a_{j+1} - a_j) - \frac{k}{3}(2c_j + c_{j+1})$ \\ \\
$d_{j} = \frac{c_{j+1} - c_{j}}{3k} $ \\

La implementación del algoritmo bicúbico consta de dos partes muy similares, ya que primero se recorre la matriz por columnas para realizar el método de splines y luego se realiza el metodo por filas para completar la matriz.
Definimos para él una clase llamada spline donde almacenaremos los arreglos $as$, $bs$, $cs$ y $ds$, estos contienen los coeficientes del polinomio para la posición $i$ de la fila o columna en donde estemos calculando los splines.

\begin{algorithm}[H]
\begin{algorithmic}[1]\parskip=1mm
\caption{void bicubico(matriz A, vector Res,int k)}
  \STATE{Para i$=0..CantFilas-1$}
  \STATE{\quad Para j$=0..CantColumnas-1$}
  \STATE{\quad \quad$ Spline spline =calcularSpline(CantColumnas,columna_j,k)$}
  \STATE{\quad Para j$=0..CantColumnas-1$}
  \STATE{\quad \quad  Para l$=0..k+1$}
  \STATE{\quad \quad \quad valor$ = spline.a[i] + spline.b[i]*l + spline.c[i]*j^2+spline.d[i]*j^3$}
  \STATE{\quad \quad \quad saturar(valor) }
  \STATE{\quad \quad \quad $Res_{i*(k+1),j*(k+1)+l}= valor$}
  \STATE{Para i$=0..CantColumnas-1$}
  \STATE{\quad Para j$=0..CantFilas-1$}
  \STATE{\quad \quad spline$ = calcularSpline(CantFilas,fila(j),k)$}
  \STATE{\quad Para j$=0..CantFilas-1$}
  \STATE{\quad \quad  Para l$=0..k+1$}
  \STATE{\quad \quad \quad valor$ = spline.a[i] + spline.b[i]*l + spline.c[i]*j^2+spline.d[i]*j^3$}
  \STATE{\quad \quad \quad saturar(valor) }
  \STATE{\quad \quad \quad $Res_{j*(k+1)+l,i)}= valor$}
\end{algorithmic}
\end{algorithm}

La función $saturar$ hace que si el valor es mayor a 255 o menor a 0 los fija en esos dos valores respectivamente.\\
Como mencionamos anteriormente se puede ver que de las lineas 1-8 se realiza el splines por columnas, luego en las lineas 9-16 se realiza el splines por filas, por lo tanto solo analizaremos el primer bloque (lineas 1- 8) para el siguiente es análogo.\\
En la linea 3 se llama al algoritmo de splines pasandole los $n$ puntos de la imagen con el cual vamos a calcular coeficientes para cada polinomio, una vez obtenido esto se recorren los $k$ puntos entre cada par de pixeles de la imagen resultante y se evalua el polinomio en ese punto, logrando asi el valor de cada punto en la imagen resultante. 
%% si se saca este new page entonces el algoritmo pasa a otra hoja porque no se divide 
\begin{algorithm}[H]
\begin{algorithmic}[1]\parskip=1mm
\caption{spline calcularSpline(int cant,arreglo(int) pixelesOriginales,int k)}
  \STATE{arreglo alfa[cantColumnas]}
  \STATE{Para j$=0..Cant-1$}
  \STATE{\quad $alfa_j =(3/k)*(pixelesOriginales_{j+1}-pixelesOriginales_{j})-(3/k)*(pixelesOriginales_{j}-pixelesOriginales_{j-1})$} 
  \STATE{arreglo(float) ln ,cn , zn }
  \STATE{l[0]=1,c[0]=0,z[0]=0}
  \STATE{Para i$=0..Cant-1$}
  \STATE{\quad $l_i = 2 * (2 * k) - k * c_{i - 1}$}
  \STATE{\quad $c_i = k / l_{i}$}
  \STATE{\quad $z_i = alfa_{i} - (k * z_{i - 1}) / l_{i}$}
  \STATE{arreglo(int) as,bs,cs,ds}
  \STATE{Para i$=0..Cant-1$}
  \STATE{\quad$ cs_{i} = z_{i} - c_{i} * cs_{i + 1} $}
  \STATE{\quad $bs_{i} = (as_{i + 1} - as{i}) / k - k * (cs_{i + 1} + 2 * cs_{i}) / 3$}
  \STATE{\quad$ (cs_{i} + 1] - cs[i]) / (3 * k)$}
  \STATE{devolver spline(as,bs,cs,ds)}
\end{algorithmic}
\end{algorithm}

En este algoritmo \cite{burden} estamos calculando los coeficientes del poliniomio dado un arreglo de elementos que van a ser nuestros puntos.\\

Este método, que podría considerarse un refinamiento del anterior, introduce la particularidad de que se le pide al polinomio interpolador que las intersecciones de las funciones que interpolan al punto $n-1$ y $n$ y al $n$ y al $n+1$, sean derivables. Como resultado, se agrega mucha más suavidad entre puntos que con la técnica anterior que solo respetaba que las funciones empiecen y terminen en el mismo punto (dando lugar a posibles picos, como en el caso de que dos puntos se interpolen con una recta ascendente y los siguientes con una descendente). Además, esta técnica evita de forma natural la oscilación del polinomio mencionada en el método anterior dado que siempre se toman polinomios por partes y, como mencionamos, al usar splines naturales podemos garantizar que la interpolación es suave en sus bordes, además de serlo en los bordes generados por los límites de cada spline. Gracias a esto, cuando se intenta reducir el error de interpolación se puede incrementar el número de partes del polinomio que se usa para construir el spline, en lugar de incrementar su grado.



\pagebreak
\section{An\'alisis}
Ahora vamos a analizar el algoritmo del PCA.
Vamos a probar el algoritmo para distintas medidas de k y α, que van a ser:
k: cantidad de vecinos a considerar en el algoritmo kNN.
α: a la cantidad de componentes principales a tomar.

Vamos a probar el algoritmo para los siguientes valores:
k: 5, 50, 500.
α: 5, 50, 100.

El "k" indica cuantas muestras voy a agarrar 

A continuación se presenta un análisis comparativo de los tres métodos implementados.
Cabe mencionar que, dado que la experimentación requería que ambas imágenes (original y modificada) tengan el mismo tamaño para poder realizar un análisis cuantitativo de los algoritmos mediante las medidas de comparación que se mencionaron en la introducción, decidimos achicar la imagen original mediante un programa de edición de imágenes para luego agrandarla mediante nuestros métodos y poder comparar los resultados obtenidos con la imagen original.

\subsection{Ventana óptima para método de Splines}
Nuestro primer análisis se encargará de encontrar un valor óptimo para la cantidad de las ventanas utilizadas en el método de splines.
Para dicho fin, elegimos correr varias instancias del método con valores de $K$ crecientes para distintos valores de ventana (4, 8, y 16 para poder comparar los resultados). Se presentan entonces los resultados obtenidos. Vale la pena destacar que no se muestra información respecto al PSNR debido a que presentaba exactamente el mismo comportamiento y no ofrecia informacion extra alguna.
\\
Como habiamos presupuesto en la introducción, agrandar el tamaño de la ventana solo hace que se tengan en cuenta valores para el punto que se quiere calcular que no depende directamente de este. 

\begin{center}
\includegraphics[scale=0.50]{imagenes/VK2.png}
\end{center}

Como puede verse a continuación, agrandar el tamaño de la ventana deja de presentar benificio alguno para valores mas altos de $K$ porque los resultados se veulven constantes:

\begin{center}
\includegraphics[scale=0.50]{imagenes/VK4.png}
\includegraphics[scale=0.50]{imagenes/VK6.png}
\includegraphics[scale=0.50]{imagenes/VK10.png}
\end{center}

A partir de este punto, el algoritmo de splines utiliza una ventana de tamaño cuatro, dado que es la que presenta mejores resultados.
De todos formas, no es cierto que siempre sea preferible una ventana más chica a una que incluya mas puntos, porque en problemas donde se quieren calcular por ejemplo trayectorias, es deseable considerar mas puntos para tener mas información.

\subsection{Análisis de los metodos}

Empleamos un análisis incremental respecto al valor de los pixeles intermedios introducidos (el valor de $k$) para poder analizar los distintos métodos de forma escalonada y presentar conclusiones mucho más claras. Dado que nuestros algoritmos solo funcionan cuando los valores de las imágenes son divisibles por $k$, no deben asumirse una correlación entre los distintos valores de este debido a que las imágenes a analizar no siempre pudieron ser las mismas.

\subsubsection{K = 2}
Nuestro primer análisis se concentra en el valor mínimo de $k$ para el cual esperabamos que el comportamiento de los tres métodos se mantenga bastante estable. Nuestra intuición proviene de la idea de que todos ellos ofrecian una perdida en la calidad de la imagen bastanta pequeña en relación al zoom pedido.
Como podemos apreciar en los gráficos a continuación, nuestra intuición se corresponde con los valores de $PSNR$, donde los tres métodos se comportan relativamente iguales, sin embargo, nos sorprende ver que para el error cuadrático medio (y para $PSNR$ también, pero en menor medida) la técnica de interpolación Bilineal obtuvo resultados muy destacables (de hecho, casi constantes), incluso frente a la técnica de Splines que esperabamos siempre tenga un mejor rendimiento. La conclusión respecto de este fenómeno se explica al final del artículo.

\begin{center}
\includegraphics[scale=0.50]{imagenes/K2PSNR.png}
\includegraphics[scale=0.50]{imagenes/K2ECM.png}
\end{center}

\subsubsection{K = 4}
Para este segundo caso es cierto que, como esperabamos, el error cuadrático medio del método de los vecinos se dispara rápidamente mientras que los de interpolación bilineal y splines se mantienen prácticamente iguales. Lo mismo sucede para los valores de PSNR, siendo los del método de vecinos los únicos que disminuyen con una diferencia de casi 5 puntos.

\begin{center}
\includegraphics[scale=0.50]{imagenes/K4PSNR.png}
\includegraphics[scale=0.50]{imagenes/K4ECM.png}
\end{center}

\subsubsection{K = 6}
Entrando en valores de $k$ mucho más elevados, nuestro análisis empezó a dejar de coincidir con lo que creiamos en un primer momento serían los resultados finales, debido a que el método de Splines no logró sacar una diferencia notoria frente al de interpolación Bilineal, sino que incluso ambos métodos se mantuvieron prácticamente constantes.  Al momento de obtener los resultados nos llamaron poderosamente la atención los valores de ECM obtenidos para las tres primeras imágenes, pero luego de hacer un análisis en conjunto de estas, llegamos a la conclusión de que la suba desmesurada en estos valores se debe a la elevada variabilidad de los contrastes en la escena (el interior de un hogar muy decorado, la foto aérea de un barrio, etc). 

\begin{center}
\includegraphics[scale=0.50]{imagenes/K6PSNR.png}
\includegraphics[scale=0.50]{imagenes/K6ECM.png}
\end{center}

\subsubsection{K = 10}
Por último, para valores que ya se consideran altos de $k$ el método de interpolación Bilineal todavía sigue desempeñandose igual o incluso a veces levemente mejor que el de Splines. Esto no solo contradice nuestra intuición, sino que debido a la performance de ambos, estos resultados colocan al método de interpolación como el más apto en relación benenficio/tiempo, muy por encima del de Splines (ambos ya muy por encima del método de vecinos a esta altura).

\begin{center}
\includegraphics[scale=0.50]{imagenes/K10PSNR.png}
\includegraphics[scale=0.50]{imagenes/K10ECM.png}
\end{center}

\subsection{Análisis de tiempos}
El análisis de tiempo, a diferencia del de los métodos, no ofreció ninguna respuesta que no hayamos podido intuir durante la codificación de los algoritmos. Es claro que a medida que el método se perfecciona en la búsqueda de resultados más suaves, también aumenta el tiempo necesario de cálculo.

\begin{center}
\includegraphics[scale=0.50]{imagenes/K2T1.png}
\includegraphics[scale=0.50]{imagenes/K2T2.png}
\includegraphics[scale=0.50]{imagenes/K4T.png}
\includegraphics[scale=0.50]{imagenes/K6T.png}
\includegraphics[scale=0.50]{imagenes/K10T.png}
\end{center}

\subsection{Analisis De Los metodos Para Imagenes Con Simbolos Alfanumericos}
En esta sección analizaremos los tres algoritmos sobre imagenes con simbolos alfanumericos. Para ellos usamos la imagen mostrada mas abajo para la cual aplicaremos los tres metodos con diferentes ks.

\begin{figure}[H]
\centering
\includegraphics[scale=0.50]{fotos/alfanum/orig.png}
\end{figure}

Primero realizamos las pruebas con el valor minimo de $k$ ($k=1$), obteniendo los siguientes resultados:

\begin{figure}[H]
    \centering
    \subfloat[Metodo 1]{{\includegraphics[width=5cm]{fotos/alfanum/k1_1.jpg} }}%
    \qquad
    \subfloat[Metodo 2]{{\includegraphics[width=5cm]{fotos/alfanum/k1_2.jpg} }}%
    \qquad
    \subfloat[Metodo 3]{{\includegraphics[width=5cm]{fotos/alfanum/k1_3.jpg} }}%
    \caption{Comparacón de metodos para $k = 1$}%
    \label{fig:example}%
\end{figure}

Como podemos ver, las tres imagenes introducen artifacts (errores visuales) que todavia no desmejoran la imagen a un nivel en el que sea imposible su comprension, por lo menos en los digitos mas externos (distinto para los numeros internos de la imagen que, debido a su tamaño inicial, ya son casi inperceptibles con este $k$ minimo). Notese como el metodo 2, como consecuencia de la nivelacion entre los valores de alto contraste del dibujo y su fondo blanco, empieza a introducir una leve 'niebla gris' alrededor de las imagenes. La misma sutiacion se plantea en el metodo tres, pero con la diferencia de que el difuminado introducido es mucho menos visible.

\\
Observemos los resultados para un $k$ un poco mas elevado ($k=2$):

\begin{figure}[H]
    \centering
    \subfloat[Metodo 1]{{\includegraphics[width=5cm]{fotos/alfanum/k2_1.jpg} }}%
    \qquad
    \subfloat[Metodo 2]{{\includegraphics[width=5cm]{fotos/alfanum/k2_2.jpg} }}%
    \qquad
    \subfloat[Metodo 3]{{\includegraphics[width=5cm]{fotos/alfanum/k2_3.jpg} }}%
    \caption{Comparacón de metodos para $k = 2$}%
    \label{fig:example}%
\end{figure}

En esta ocacion, el comportamiento sigue los lineamientos generales del caso anterior, con la salvedad de que ninguna de las tres imagenes ya es comprensible. Vemos como la 'niebla' comentada en el caso anterior avanzo rapido en el metodo dos, para casi difuminar la imagen por completo. En el caso del metodo de splines (el tercero)se puede empezar a ver un pequeño sombreado alrededor de los bordes de los elementos en la imagen, pero a diferencia del metodo dos, esta solo se extiende a las cercanias y no avanza por toda la imagen.

\\
Por ultimo, presentamos los resultados para $k=4$:

\begin{figure}[H]
    \centering
    \subfloat[Metodo 1]{{\includegraphics[width=5cm]{fotos/alfanum/k5_1.jpg} }}%
    \qquad
    \subfloat[Metodo 2]{{\includegraphics[width=5cm]{fotos/alfanum/k5_2.jpg} }}%
    \qquad
    \subfloat[Metodo 3]{{\includegraphics[width=5cm]{fotos/alfanum/k5_3.jpg} }}%
    \caption{Comparacón de metodos para $k = 4$}%
    \label{fig:example}%
\end{figure}

Como era de esperarse las tres imagenes resultantes ya perdieron comprension en su totalidad. Ademas, el segundo y tercer metodo, presentan una alta cantidad de ruido por el difuminado producido respecto de la imagen original.
Queda entonces a la vista una caracteristica que no estabamos considerando hasta entonces en nuestro analisis. El metodo de los vecinos puede llegar a ofrecer resultados favorables si se cumplen algunas caracteristicas deseables (nuestra intuición preveia que este metodo seria superado por los anteriores en cualquier situacion) como en este caso. El alto contraste entre las imagenes, hace que en los metodos que introducen cierta correlación o suavizado entre pixeles se genere un sombreado que hace mas borrosas las imagenes. En contra de nuestros pronosticos, el metodo de los vecinos podria ser un excelente candidato en estos casos.


\subsection{Analisis De Los metodos Para Paisajes}

En esta sección analizamos como se comportan los metodos para fotos de paisajes. Tomamos la siguiente foto:

\begin{figure}[H]
\centering
\includegraphics[scale=0.50]{fotos/paisaje/orig.png}
\end{figure}

Primero lo hacemos para $k=1$, se obtiene esto:

\begin{figure}[H]
    \centering
    \subfloat[Metodo 1]{{\includegraphics[width=5cm]{fotos/paisaje/k1_1.png} }}%
    \qquad
    \subfloat[Metodo 2]{{\includegraphics[width=5cm]{fotos/paisaje/k1_2.png} }}%
    \qquad
    \subfloat[Metodo 3]{{\includegraphics[width=5cm]{fotos/paisaje/k1_3.png} }}%
    \caption{Comparacón de metodos para $k = 1$}%
    \label{fig:example}%
\end{figure}

Los artifact (Errores visuales) que vemos aquí son, bla bla bla (COMPLETAR!)
\\
Ahora lo hacemos para $k=2$, se obtiene esto:

\begin{figure}[H]
    \centering
    \subfloat[Metodo 1]{{\includegraphics[width=5cm]{fotos/paisaje/k2_1.png} }}%
    \qquad
    \subfloat[Metodo 2]{{\includegraphics[width=5cm]{fotos/paisaje/k2_2.png} }}%
    \qquad
    \subfloat[Metodo 3]{{\includegraphics[width=5cm]{fotos/paisaje/k2_3.png} }}%
    \caption{Comparacón de metodos para $k = 2$}%
    \label{fig:example}%
\end{figure}

Los artifact que vemos aquí son, bla bla bla (COMPLETAR!)
\\
Ahora lo hacemos para $k=3$, se obtiene esto:

\begin{figure}[H]
    \centering
    \subfloat[Metodo 1]{{\includegraphics[width=5cm]{fotos/paisaje/k3_1.png} }}%
    \qquad
    \subfloat[Metodo 2]{{\includegraphics[width=5cm]{fotos/paisaje/k3_2.png} }}%
    \qquad
    \subfloat[Metodo 3]{{\includegraphics[width=5cm]{fotos/paisaje/k3_3.png} }}%
    \caption{Comparacón de metodos para $k = 3$}%
    \label{fig:example}%
\end{figure}

\subsection{Analisis De Los metodos para rostros}

En esta sección analizamos como se comportan los metodos para fotos de rostros. Tomamos la siguiente foto:

\begin{figure}[H]
\centering
\includegraphics[scale=0.50]{fotos/rostro/orig.png}
\end{figure}

Primero lo hacemos para $k=1$, se obtiene esto:

\begin{figure}[H]
    \centering
    \subfloat[Metodo 1]{{\includegraphics[width=5cm]{fotos/rostro/k1_1.png} }}%
    \qquad
    \subfloat[Metodo 2]{{\includegraphics[width=5cm]{fotos/rostro/k1_2.png} }}%
    \qquad
    \subfloat[Metodo 3]{{\includegraphics[width=5cm]{fotos/rostro/k1_3.png} }}%
    \caption{Comparacón de metodos para $k = 1$}%
    \label{fig:example}%
\end{figure}

Los artifact (Errores visuales) que vemos aquí son, bla bla bla (COMPLETAR!)
\\
Ahora lo hacemos para $k=2$, se obtiene esto:

\begin{figure}[H]
    \centering
    \subfloat[Metodo 1]{{\includegraphics[width=5cm]{fotos/rostro/k2_1.png} }}%
    \qquad
    \subfloat[Metodo 2]{{\includegraphics[width=5cm]{fotos/rostro/k2_2.png} }}%
    \qquad
    \subfloat[Metodo 3]{{\includegraphics[width=5cm]{fotos/rostro/k2_3.png} }}%
    \caption{Comparacón de metodos para $k = 2$}%
    \label{fig:example}%
\end{figure}

Los artifact que vemos aquí son, bla bla bla (COMPLETAR!)
\\
Ahora lo hacemos para $k=3$, se obtiene esto:

\begin{figure}[H]
    \centering
    \subfloat[Metodo 1]{{\includegraphics[width=5cm]{fotos/rostro/k3_1.png} }}%
    \qquad
    \subfloat[Metodo 2]{{\includegraphics[width=5cm]{fotos/rostro/k3_2.png} }}%
    \qquad
    \subfloat[Metodo 3]{{\includegraphics[width=5cm]{fotos/rostro/k3_3.png} }}%
    \caption{Comparacón de metodos para $k = 3$}%
    \label{fig:example}%
\end{figure}

\pagebreak
\section{Resultados}

\subsection{Análisis de tiempos en función de los parámetros de entrada}
En esta sección analizaremos de manera experimental como varían los tiempos de ejecución de los algoritmos descriptos al variar el largo y el ancho de la matriz y la cantidad de sanguijuelas del sistema.

\subsubsection{Ancho en función del tiempo}
Para comenzar, tomaremos un parabrisas con 50 sanguijuelas tal que estas solo afectan un punto de la discretización, y para una granularidad fija de $1.0$ iremos variando el largo del parabrisas. De esta manera, comenzaremos con un parabrisas de $50 \times 50$ luego uno de $60 \times 50$ y así aumentando de manera lineal ambos parámetros hasta llegar a un parabrisas de $100 \times 50$. Resolveremos cada uno de estos sistemas utilizando ambos métodos implementados (Gauss y Descomposición LU). Los resultados obtenidos pueden verse en el siguiente gráfico:

\begin{center}
 \includegraphics[width=400pt]{imagenes/testeo/anchoGauss.png}
\end{center}

Sabemos que al aumentar el largo del parabrisas de manera lineal, aumentará también de manera lineal el número de ecuaciones en nuestra matriz de resolución. Lo que puede observarse en este gráfico es que con un aumento lineal del largo del parabrisas, el tiempo de ejecución aumenta de manera semi-lineal. Esto era esperable ya que sabemos que tanto la Eliminación Gaussiana como la Descomposición LU tienen una complejidad igual a $O(n*p^2)$, y dado que en nuestro modelo utilizamos el largo del parabrisas para definir el tamaño de la banda en la matriz de resolución (es decir $p$), resulta lógico que al aumentar el largo, se obtuviera un aumento casi cuadrático en el tiempo de ejecución.
\\
Además, en este gráfico se pueden observar que tanto los tiempos de la factorización LU como la de Eliminación GAussiana son similares.
\\
Ahora, utilizando la misma familia de parabrisas descrita anteriormente, veremos cómo se comportan ambos métodos de salvación.
\\
Dado que nos aseguramos que cada sanguijuela solo toque un punto de la discretización, nos aseguramos que para cada una de las sanguijuelas podremos utilizar el método de Sherman-Morrison. Para estos algoritmos, el gráfico es el siguiente:
\\
\begin{center}
 \includegraphics[width=400pt]{imagenes/testeo/anchoSalv.png}
\end{center}

Como era de esperar, el algoritmo que utiliza la optimización de Sherman-Morrison es mucho más rápido y escala mejor que la versión simple que debe calcular todo el sistema desde cero, ademas el Sherman-Morrison realiza operaciones sobre vectores y escalares haciendo la diferencia en el modo de comparación respecto al método anterior.

\subsubsection{Largo en función del tiempo}
Ahora analizaremos que sucede dejando fijo el ancho y variando el largo del parabrisas. Las condiciones son las mismas que en el test anterior, solo que ahora el ancho permanece constante igual a $50$ y se varía el largo de $50$ a $100$.

\begin{center}
 \includegraphics[width=400pt]{imagenes/testeo/largoGauss.png}
\end{center}

En este caso los tiempos de ejecución crecen de manera estrictamente lineal. Esto se debe que a diferencia del ancho, el largo no interviene en el cálculo del tamaño de la banda de la matriz de resolución. Luego, al aumentar el largo, solo aumenta la cantidad de incógnitas $n$.
\\
Aplicando el mismo experimento para los dos métodos de salvación:

\begin{center}
 \includegraphics[width=400pt]{imagenes/testeo/largoSalv.png}
\end{center}

Al igual que lo dicho en la sección anterior, aquí también se pueden ver las ventajas de utilizar Sherman-Morrison. Además en este caso también puede apreciarse el comportamiento lineal de ambos algoritmos al variar el largo del sistema.

\subsubsection{Cantidad de sanguijuelas en función del tiempo}
Para el siguiente experimento, variamos la cantidad de sanguijuelas y dejamos fija tanto la granularidad como el largo y el ancho del parabrisas. Nuevamente, por una cuestión de simplicidad, las sanguijuelas solo afectan un punto de la discretización. Para el experimento tomamos un parabrisas de $100 \times 100$, una granularidad igual a $1.0$, y variamos la cantidad de sanguijuelas desde $10$ hasta $100$. Resolviendo el sistema con el algoritmo de Gauss y Descomposición LU, se obtuvo el siguiente grafico.

\begin{center}
 \includegraphics[width=400pt]{imagenes/testeo/sangGauss.png}
\end{center}

Como vemos, no se muestra ningún patrón visible al modificar la cantidad de sanguijuelas del sistema. Esto se debe a que la cantidad de incógnitas continúa siendo la misma.
\\
Ahora, utilizando el mismo experimento para el problema del último disparo, resolviendo este parabrisas con ambos algoritmos, se obtuvo este gráfico:

\begin{center}
 \includegraphics[width=400pt]{imagenes/testeo/sangSalv.png}
\end{center}

En este gráfico se puede apreciar cómo si bien ambos algoritmos dependen de manera lineal de la cantidad de sanguijuelas, el algoritmo de Sherman-Morrison presenta una mejor velocidad y una mejor escalabilidad.

\subsubsection{Granularidad en función del tiempo}
Por último, veremos como afecta variar la granularidad de la discretización para ver como se ve afectada la performance. Para este experimento se dejan fijos el largo y el ancho, iguales a $100$, la cantidad de sanguijuelas iguales a $5$ y se varía la granularidad desde $0.4$ hasta $0.9$, aumentando de $0.1$ en cada paso. Se obtiene lo siguiente:

\begin{center}
 \includegraphics[width=400pt]{imagenes/testeo/granuGauss.png}
\end{center}

En el gráfico puede observarse que una disminución lineal de la granularidad produce un aumento cuadrático en el tiempo de ejecución. Esto se debe a que tanto la cantidad de filas y la cantidad de columnas viene dado por el largo/ancho del parabrisas, dividido por granularidad. Dado que el tamaño de nuestra matriz de resolución del problema viene dado por $\text{Cantidad De filas} \times \text{Cantidad De Columnas}$ esto será lo mismo que  $(\text{Largo} \times \text{Ancho}) / \text{granularidad}^2$. En esta fórmula puede verse claramente que disminuir la granularidad de manera lineal produce un aumento cuadrático en el número de incógnitas de nuestro problema.

\subsection{Análisis de temperatura en función de las discretizaciones}
El siguiente analisis que vamos a realizar sobre una instancia de 100 de largo por 100 de ancho con una sola sanguijuela en el punto $(45 45)$ de radio 1 con el fin de observar como afecta la granularidad al resultado del sistema y al punto crítico.
\\
Lo que esperamos observar es que para una granularidad mayor, el resultado posea un mayor grado de error, disminuyendo la precisión de la solución mientras que para granularidades menores el resultado se aproxime más al modelo continuo.
\\
Para ello tomamos 4 diferentes granularidades: $10, 5, 1$, y $0.5$.
\\
En el caso de una granularidad de $10$ la sanguijuela no podrá ser representada en ninguna de las ecuaciones del sistema, ya que no esta en contacto con ninguno de los puntos de la discretización. En este caso la respuesta arrojada por el algoritmo es la siguiente:
\begin{figure}[H]
\centering
\includegraphics[width=400pt]{otrostests/11.png}    
\caption{ejecucion de 1sanguijuela10.in}
\end{figure}
Lo que intuitivamente sabemos es que no es correcto, ya que explícitamente agregamos una sanguijuela con una temperatura muy considerable. Sin embargo la temperatura del punto critico que el sistema calcula es de $-100$, por lo que uno esperaría encontrar en un sistema sin sanguijuelas.
\\
Para la granularidad de 5 ya es posible caracterizar a la sanguijuela en una de las ecuaciones, por lo que el gráfico obtenido será el siguiente:
\begin{figure}[H]
\centering

\includegraphics[width=400pt]{otrostests/12.png}
\caption{ejecución de 1sanguijuela5.in}

\end{figure}
\\
De esta manera podemos intuir que la respuesta es más próxima a la verdadera, ahora la temperatura del punto crítico es de $256.124$.
\\
Continuamos disminuyendo la granularidad, ahora con granularidad 1 se obtiene la siguiente respuesta:
\begin{figure}[H]
\centering

\includegraphics[width=400pt]{otrostests/13.png}
\caption{ejecución de 1sanguijuela1.in}

\end{figure}

Para esta respuesta ya pareciera que los resultados son más precisos, ya puede observarse cierta continuidad en la solución. Además otro hecho destacable es que la temperatura del punto crítico ahora es $333.008$, esto es un $30 \%$ mayor que para la granularidad anterior.
\\
Por último realizamos el mismo sistema pero para una granularidad de 0.1 obteniendo el siguente gráfico:

\begin{figure}[H]
\centering
\includegraphics[width=400pt]{otrostests/14.png}
\caption{ejecución de 1sanguijuela0\_5.in}
 
\end{figure}

\\
Ahora ya es posible ver claramente la 'continuidad' de la solución. La temperatura del punto crítico es 335.65, osea un $0.7 \%$ mayor con respecto a la solución anterior.
\\
Puede notarse a simple vista que una granularidad más pequeña produce resultados que aproximan más exactamente a lo que uno creería es el resultado real de la ecuación diferencial de la que partimos en la introducción, ya que para una granularidad de $5$ pueden verse grandes bloques discretos, pero para la discretización de $0.1$ puede verse que estos bloques desaparecen y puede notarse una cierta 'suavidad' en el resultado obtenido, propio de una función continua y derivable. Esto puede verse en el gráfico ya que la tonalidad va oscureciendo hasta llegar a rojo generando un círculo grande, de ser chico estaríamos viendo un pico en la función y no sería derivable.
\\
Además, al disminuir la granularidad, está claro que la cantidad de incógnitas cercanas al punto crítico aumenta, por lo que, en caso de no poder tomarlo de manera exacta, al menos podremos tomar un vecino muy próximo a este por lo que tendemos a pensar que la presición del resultado también aumentará por ese lado.
Por último creamos 3 instancias del problema para ejecutarlas con dos granularidades distintas, uno y cuatro, para poder comparar mejor que pasaba con el punto crítico decidimos hacerlas de distintas medidas y sanguijuelas distribuidas de distinta forma para poder observar que ocurría.
Para el primer test, lo que se realizó fue una matriz cuadrada con 5 sanguijuelas las cuales estaban concentradas en la esquina izquierda inferior y otra en la esquina derecha superior, luego de correr Sherman-Morrison quitó la de la derecha superior que era la que tenía mayor temperatura y arrojó los siguientes gráficos de calor.

\begin{figure}[H]
 \centering
 \begin{subfigure}
  	
  	\includegraphics[width=300px]{imagenes/testpropios/test11.png}
  	\caption{ejecución test1.in}
 \end{subfigure}
 
 \begin{subfigure}
   	\includegraphics[width=300px]{imagenes/testpropios/test14.png}
    \caption{ejecución con granularidad 4, archivo test1b.in }
 \end{subfigure}
 
\end{figure}


Como se puede ver en este caso, y como mencionamos anteriormente a pesar de que las sanguijuelas estan en la esquina izquierda inferior el punto crítico se ve más afectado cuando la granularidad es mayor, dándonos en este caso una respuesta que no es correcta para el punto crítico ya que la imagen con menor granularidad en el punto crítico tiene menor temperatura, sabiendo que este último tiene menor precisión. 

Para el segundo se agranda el parabrisas y se agregan sanguijuelas, corremos el mismo test para los mismo valores de granularidad, uno y cuatro y obtuvimos el siguiente resultado.

\begin{figure}[H]
 \centering
   \begin{subfigure}
  		\includegraphics[width=300px]{imagenes/testpropios/test21.png}
    	\caption{ejecución test2.in}
   \end{subfigure}
  
   \begin{subfigure}
  		\includegraphics[width=300px]{imagenes/testpropios/test24.png}
    	\caption{ejecución test2b.in}
   \end{subfigure}

\end{figure}

A pesar que las sanguijuelas estan más repartidas volvemos a notar lo mismo que se veía en el primer test, a mayor granularidad mayor afectado se ve el punto crítico, a pesar que las sanguijuelas estan más alejadas.  

Por último alargamos más el parabrisas y corrimos el test para ver que era lo que sucedía, aunque como se comportaron los casos anteriores se espera que se repita el patrón, que cuanto mas granularidad más afectado se vea el punto crítico.
  \begin{figure}[H]
  \centering
 	\begin{subfigure}
  		\includegraphics[width=300px]{imagenes/testpropios/test31.png}
    	\caption{ejecución test3.in}
	\end{subfigure}
	
 	\begin{subfigure}
 		\includegraphics[width=300px]{imagenes/testpropios/test34.png}
    	\caption{ejecución test3b.in}
  \end{subfigure}

\end{figure}
Luego de realizar los tres test y notar que se repite el patron, de forma más particular en el ultimo caso ya que las sanguijuelas afectan las temperaturas haciendo que aumente la temperatura entre medio de ellas afectando el punto crítico.
Con lo cual podemos asegurar que mediante la granularidad sea más grande la precisión va a ser menor y el punto crítico va a ser mas afectados, y esto podria ser de suma importancia al momento de quitar la sanguijuela ya que podria afectar al punto critico al punto de remover otra sanguijuela porque su temperatura sobrepasa el límite.

\pagebreak
\section{Conclusiones}

Como se mencionó con anterioridad, nuestra intuición relacionaba fuertemente a los métodos en cuanto a calidad/desempeño.
Como se pudo ver a lo largo del análisis realizado, el método de splines nunca logró sacar una diferencia significativa respecto al método de interpolación bilineal. También se puede apreciar como este último tiene un mejor desempeño temporal en todos los casos.
Esto coloca a la interpolación bilineal como la mejor opción en cuanto a tiempo y calidad, dado que la ganancia por splines es minima respecto al tiempo extra. En un momento $(K = 2)$, nos llamó poderosamente la atención que el método bilineal haya obtenido un mejor resultado que el método de splines, pero teniendo en cuenta como se terminaron comportando ambos métodos a lo largo de todo el análisis, ahora ya no parece un resultado tan anómalo. Sin embargo, no logramos llegar a una conclusión que justifique el porque de una diferencia tan significativa.

En cuanto al método de los vecinos, como bien dijimos al principio, se comporta relativamente bien para valores de $k$ mínimos, pero enseguida que este crece, el método pierde fiabilidad.


\pagebreak
\section{Apendice}

\subsection{Compilación y formato de ejecución del programa}

\subsubsection{Compilación}

Como se encuentra mencionado en el archivo $README.txt$, en la carpeta $src$ de la carpeta raíz del trabajo se encuentra un $Makefile$. Así es que ejecutando el comando

$\texttt  make $

se genera el ejecutable $tp$.

\subsubsection{Formato de ejecución}

El comando para correr el experimento para una instancia es: 

$\texttt ./tp archivoEntrada metodo k $

Donde:

\begin{itemize}
  \item $archivoEntrada$ es la imagen original
  \item $metodo$ es un entero entre 0 y 2 donde 
	
  \begin{itemize}
    \item 0 es el metodo de vecinos
    \item 1 es el metodo Bilineal
    \item 2 es el metodo de Splines
  \end{itemize}

  \item $k$ es un entero que indica cuanto zoom se hará en la foto
\end{itemize}


\end{document}
