\subsection{Ventajas en el uso de una matriz banda}

Una particularidad del sistema de ecuaciones planteado es que el valor de cada elemento del vector de resoluci\'on depende exclusivamente de, como m\'aximo, otros 4 valores que adem\'as son \"cercanos\" al elemento a resolver, llamando cercano a un elemento que se encuentra a no m\'as de una fila de distancia, dejando una gran cantidad de elementos (la gran mayor\'ia en 0). Este problema en particular plantea una matriz banda con una cantidad de elementos no nulos mucho menor a aquellos que tienen un valor no nulo. Siendo m\'as precisos la matriz tiene $(n \times m) \times (n \times m)$ elementos, con menos de $5 \times (m \times n)$ elementos no nulos. Gracias a eso es que pudimos realocar los valores en un espacio f\'isico muy por debajo de lo que la representaci\'on matricial de un caso general requiere, mapeando los valores distintos de 0 a la matriz f\'isica y los valores nulos siendo entregados sin ser alojados en ninguna matriz.

\subsection{Discretizaci\'on de un problema continuo}

Durante la resoluci\'on del problema pudimos ver que cambiando el $h$, es decir, la medida de discretizaci\'on del continuo del parabrisas, obtenemos diferentes grados de error. A mayor $h$ esperamos un error m\'as pronunciado.
Una posible definici\'on de $Probabilidad$ es el nivel de conocimiento sobre el sistema. An\'alogamente podemos decir que el nivel de discretizaci\'on de un sistema continuo nos da una idea del conocimiento sobre el mismo.

Por otro lado, una mayor discretizaci\'on del sistema impacta directamente sobre los tiempos de c\'omputo en su resoluci\'on. Queda entonces a criterio del desarrollador valorar ambos par\'ametros y balancear entre soluciones m\'as precisas y tiempos de resoluci\'on m\'as cortos. Esta elecci\'on est\'a condicionada por los problemas acarreados provinientes de la resoluci\'on a trav\'es de computadoras, que no tienen la posibilidad de representar todos los n\'umeros de la recta real, generando problemas de precisi\'on.

\subsection{P\'erdida de precisi\'on en el uso de aritm\'etica de punto flotante}

As\'i como la separaci\'on de los puntos del sistema queda, hasta cierto punto, a criterio de quien resuelve el problema, existe una gran limitaci\'on debido	a la representaci\'on de los n\'umeros de punto flotante en la computadora. Si bien C++ es independiente a la arquitectura, las arquitecturas modernas suelen utilizar el estandar IEEE754 para representar la recta num\'erica. Sabemos que realizando operaciones de punto flotante perdemos precisi\'on, si bien esta puede ser calculada y acotada. 

Entonces es as\'i como la resoluci\'on del problema se ve afectado no solo por la reducci\'on de un subespacio $\mathbb{R}2$ en un conjunto finito de puntos, sino tambi\'en por la impresici\'on propia de la computadora al realizar los c\'omputos.

\subsection{Otras aplicaciones}

Vemos como este tipo de problema donde el valor de cada elemento de un espacio en 2 dimensiones (o m\'as) depende del valor de elementos cercanos puede aplicarse en diversas \'areas, por ejemplo en el procesamiento de im\'agenes, donde el suavizado en el zoom de una im\'agen se puede realizar a trav\'es del promedio de los vecinos de cada pixel. Tambi\'en es de posible aplicaci\'on para la propagaci\'on de ondas en distintos medios. Es interesante ver tambi\'en como algunas t\'ecnicas gen\'ericas ayudan a la velocidad de c\'omputo y luego se pueden realizar optimizaciones dentro del dominio de cada sistema en particular, como fue en nuestro caso Sherman-Morrison para casos donde el sistema var\'ia levemente.
