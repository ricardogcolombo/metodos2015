\subsection{Complejidades}
En el apartado anterior pudimos comprobar que las compliejidades teoricas de los algoritmos eran las esperadas. 
\\
Ademas vimos que los tiempos experimentaes de la descomposición LU eran muy similares a los tiempos del metodo de gauss, lo que era esperable ya que ambos metodos tenían la misma complegidad y eran en espiritu muy similares.
\\
Por otra parte observamos que el metodo de sherman morrison, cuando era posible aplicarlo, escalaba mejor que el metodo de salvación simple, lo cual tambien era esperable de acuerdo a la complejidad teorica calculada.
\subsection{Complejidades}
\Completar