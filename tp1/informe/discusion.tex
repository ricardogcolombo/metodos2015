\subsection{Complejidades}
En el apartado anterior pudimos comprobar que las complejidades teóricas de los algoritmos eran las esperadas 
\\
Además vimos que los tiempos experimentales de la descomposición LU eran muy similares a los tiempos del método de gauss, lo que era esperable ya que ambos métodos tenían la misma complejidad y eran en espíritu muy similares.
\\
Por otra parte observamos que el método de Sherman-Morrison, cuando era lógico aplicarlo, escalaba mejor que el método de salvación simple, lo cual también era esperable de acuerdo a la complejidad teórica calculada.
\subsection{Complejidades}
\Completar
