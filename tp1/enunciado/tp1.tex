\documentclass[11pt, a4paper]{article}
\usepackage[utf8]{inputenc}
\usepackage[spanish]{babel}
\usepackage{a4wide,amsmath,amsfonts,graphicx,color,todonotes}
\parskip = 11pt
\newcommand{\real}{\mathbb{R}}
\newcommand{\nat}{\mathbb{N}}

\newcommand{\atacante}{sanguijuela}
\newcommand{\capitan}{Capit\'an Guybrush Threepwood}
\newcommand{\objeto}{parabrisas}
\newcommand{\nave}{El Pepino Marino}
\newcommand{\titulotp}{``No creo que a \'el le gustar\'a eso''}

\newcommand{\revJ}[1]{{\color{red} #1}}

\addtolength{\topmargin}{-1cm}
\addtolength{\textheight}{2.2cm}
\addtolength{\textwidth}{-.5cm}

\begin{document}
\begin{centering}
\large\bf Laboratorio de M\'etodos Num\'ericos - Primer Cuatrimestre 2015 \\
\large\bf Trabajo Pr\'actico N\'umero 1: \emph{\titulotp}\\
\end{centering}

\vskip 0.5 cm
\hrule
\vskip 0.1 cm

{\noindent \bf Introducci\'on}

El afamado \capitan\ se encuentra nuevamente en problemas. El
\objeto\ de su nave \nave\ est\'a siendo atacado simult\'aneamente por varios
dispositivos hostiles vulgarmente conocidos como \emph{\atacante s
mutantes}. Estos artefactos se adhieren a los \objeto\ de las naves y
llevan a cabo su ataque aplicando altas temperaturas sobre la superficie, con
el objetivo de debilitar la resistencia del mismo y permitir as\'\i \
un ataque m\'as mort\'\i fero. Cada \atacante\ consta de una \emph{sopapa
de ataque} circular, que se adhiere al \objeto\ y aplica una temperatura
constante sobre todos los puntos del \objeto\ en contacto con la sopapa.

Para con\-tra\-rres\-tar estas acciones hostiles, el \capitan\ cuenta con el sistema de refrigeraci\'on de la nave, que puede aplicar una temperatura constante de -100${}^o$C a los bordes del \objeto.
El manual del usuario de la nave dice que si el punto central del \objeto\ 
alcanza una temperatura de 235${}^o$C, el \objeto\ no resiste la temperatura
y se destruye. Llamamos a este punto el \emph{punto cr\'\i tico} del
\objeto.

En caso de que el sistema de refrigeraci\'on no sea suficiente para salvar
el punto cr\'\i tico, nues\-tro \capitan\ tiene todav\'\i a una posibilidad
adicional: puede des\-tru\-ir alguna de las \atacante s. Es importante destacar que solo puede destruir una de ellas, ya que la eliminaci\'on de muchas \atacante s consumir\'ia  energ\'ia vital para la finalizaci\'on  de la misi\'on, y llevar\'ia \'unicamente al fracaso de la misma.
La situaci\'on es desesperante, y nuestro h\'eroe debe tomar una r\'apida
determinaci\'on: debe decidir, cuando sea posible, qu\'e \atacante\ eliminar de modo tal que el \objeto\ resista hasta alcanzar la base m\'as cercana.

{\noindent \bf El modelo}

Suponemos que el \objeto\ es una placa rectangular de $a$ metros de ancho y $b$ metros de altura. Llamemos $T(x,y)$ a la temperatura en el punto dado por las coordenadas $(x,y)$. En el estado estacionario, esta temperatura satisface la ecuaci\'on del calor:

\begin{equation}\label{eq:calor}
\frac{\partial^2T(x,y)}{\partial x^{2}}+\frac{\partial^2 T(x,y)}{\partial y^{2}} = 0.
\end{equation}

\noindent La temperatura constante en los bordes queda definida por la siguiente ecuaci\'on:
\begin{equation}
T(x,y) = -100^o\textrm{C}~~~~~\textrm{si } x = 0,a \textrm{ \'o } y = 0,b.
\label{eq:borde}
\end{equation}

\noindent De forma an\'aloga es posible fijar la temperatura en aquellos puntos cubiertos por una \atacante, considerando $T_s$ a la temperatura ejercida por las mismas.

El problema en derivadas parciales dado por la primera ecuaci\'on con las condiciones de contorno presentadas recientemente, permite encontrar la funci\'on $T$ de temperatura en el \objeto, en funci\'on de los datos mencionados en esta secci\'on.

%(x_j,y_i) = (ib/n,ja/m)
Para estimar la temperatura computacionalmente, con\-si\-de\-ra\-mos la siguiente discretizaci\'on del \objeto: sea $h \in \mathbb{R}$ la granularidad de la discretizaci\'on, de forma tal que $a = m\times h$ y $b = n \times h$, con $n,m \in \mathbb{N}$, obteniendo as\'i una grilla de $(n+1)\times(m+1)$ puntos. Luego, para $i=0,1,\dots,n$ y $j=0,1,\dots,m$, llamemos $t_{ij} = T(x_j,y_i)$ al valor (desconocido) de la funci\'on $T$ en el punto $(x_j, y_i) = (ih, jh)$, donde el punto $(0,0)$ se corresponde con el extremo inferior izquierdo del \objeto.
La aproximaci\'on por \emph{diferencias finitas} de la ecuaci\'on del calor afirma que:
\begin{equation}
t_{ij} \ =\ \frac{ t_{i-1,j} + t_{i+1,j} + t_{i,j-1} + t_{i,j+1}}{4}.\label{eq:calordd}
\end{equation}

Es decir, la temperatura de cada punto de la grilla debe ser igual al promedio de las tem\-pe\-ra\-tu\-ras de sus puntos vecinos en la grilla. Adicionalmente, conocemos la temperatura en los bordes, y los datos del problema permiten conocer la temperatura en los puntos que est\'an en contacto con las \atacante s.

{\noindent \bf Enunciado}

Se debe implementar un programa en \verb+C+ o \verb-C++- que tome como entrada los par\'ametros del problema ($a$, $b$, $h$, junto con las posiciones, radio y temperatura de las \atacante s) y calcule la temperatura en el \objeto\ utilizando el modelo propuesto en la secci\'on anterior.
 % y que determine a qu\'e \atacante s dispararle con el fin de evitar que se destruya el \objeto. El m\'etodo para determinar las \atacante s que ser\'an destru\'idas queda a criterio del grupo, y puede ser exacto o heur\'istico.

Para resolver este problema, se deber\'a formular un sistema de ecuaciones lineales que permita calcular la temperatura en todos los puntos de la grilla que discretiza el \objeto, e implementar el m\'etodo de Eliminaci\'on Gaussiana (EG) para resolver este sistema particular. Dependiendo de la granularidad utilizada en la discretizaci\'on, el sistema de ecuaciones resultante para este problema puede ser muy grande. Luego, es importante plantear el sistema de ecuaciones de forma tal que posea cierta estructura (i.e., una matriz banda), con el objetivo de explotar esta caracter\'istica tanto desde la \emph{complejidad espacial} como \emph{temporal} del algoritmo. Adem\'as de la estructura banda, en determinados casos es posible utilizar la descomposici\'on LU del sistema para acelerar el cómputo de evaluar qu\'e suceder\'ia si una \atacante\ fuese eliminada.

En funci\'on de la implementaci\'on, como m\'inimo se pide:
\begin{enumerate}
% \item \textit{Baseline:} Representar la matriz del sistema utilizando como estructura de datos los tradicionales arreglos bi-dimensionales, e implementar el algoritmo cl\'asico de EG. \label{enum:EGcomun}
\item \textit{Explotando la estructura:} Representar la matriz del sistema aprovechando la estructura banda de la misma, haciendo hincapi\'e en la complejidad espacial. Realizar las modificaciones necesarias del algoritmo de EG clásico para que aproveche la estructura banda de la matriz, e implementarlo. \label{enum:EGbanda}
\item \textit{Factorización LU:} Implementar un algoritmo que permita calcular la descomposición LU de la matriz, aprovechando la estructura banda de la misma. Además, se deben implementar los algoritmos de \textit{forward}/\textit{back substitution} para resolver el sistema utilizando la factorización LU. Discutir alternativas de implementación buscando minimizar el espacio utilizado. \label{enum:LU}
\item \textit{Última Esperanza:} Implementar un algoritmo para decidir si es posible salvar el \objeto\ de la destrucción mediante la eliminaci\'on de \emph{una} \atacante. En caso de ser posible, el algoritmo debe determinar la \emph{mejor} \atacante\ a eliminar, (i.e., aquella que al ser removida genera la menor temperatura en el punto cr\'itico). \label{enum:Esperanza}
\item \textit{No hay tiempo que perder:} Existen casos donde la eliminación de una sanguijuela modifica \textit{levemente} el sistema, alterando solo una \emph{fila} de este. En estos casos, es posible utilizar la factorizaci\'on LU y lo que se conoce como fórmula de Sherman–Morrison \cite{ShermanMorrison}:\\
\begin{equation}
	(A+ uv^t)^{-1} \ =\ A^{-1} - \frac{ A^{-1} u v^t A^{-1} }{1+v^t A^{-1}u}.\label{eq:sm}
\end{equation}
para acelerar el cómputo de evaluar la temperatura de eliminar una \atacante. Identificar estos casos, analizar cómo se modifica el sistema, e implementar una versión del algoritmo propuesto en \ref{enum:Esperanza} que explote esta propiedad cuando sea posible. \label{enum:SM}
\end{enumerate}
\vspace*{-0.3cm}
En funci\'on de la experimentaci\'on, como m\'inimo debe realizarse lo siguiente:
\begin{itemize}
\item Considerar al menos tres instancias originales de prueba, generando discretizaciones variando la granularidad para cada una de ellas y comparando el valor de la temperatura en el punto cr\'itico. Se sugiere presentar gr\'aficos de temperatura para los mismos, ya sea utilizando las herramientas provistas por la c\'atedra o implementando sus propias herramientas de visualizaci\'on.
\item Analizar el tiempo de c\'omputo requerido en funci\'on de la granularidad de la discretizaci\'on, buscando un compromiso entre la calidad de la soluci\'on obtenida y el tiempo de c\'omputo requerido. Comparar los resultados obtenidos para alguna de las variantes propuestas en \ref{enum:EGbanda} y \ref{enum:LU}, y analizar ventajas y desventajas de ambos esquemas. ?`Cómo impacta en los resultados la elección del esquema \ref{enum:EGbanda} y \ref{enum:LU}? ?`Por qué?
\item Estudiar el comportamiento del m\'etodo propuesto para la estimaci\'on de la temperatura en el punto cr\'itico y para la eliminaci\'on de \atacante s, comparando las variantes propuestas en \ref{enum:Esperanza} y \ref{enum:SM}.
\end{itemize}

Finalmente, se deber\'a presentar un informe que incluya una descripci\'on detallada de los desarrollos teóricos propuestos, m\'etodos implementados y las decisiones tomadas, incluyendo las estructuras utilizadas para representar la matriz banda  y los experimentos realizados, junto con el correspondiente an\'alisis y siguiendo las pautas definidas en el archivo \verb+pautas.pdf+.

\textit{Opcional}:
\vspace*{-0.3cm}
\begin{itemize}
	\item Al aplicar Eliminación Gaussiana sobre el sistema asociado a este problema en particular, ?`Es necesario pivotear? ?`Por qu\'e?
	\item ?`Cómo modificaría el algoritmo propuesto en \ref{enum:Esperanza} para aprovechar la fórmula de Sherman–Morrison \cite{ShermanMorrison} para cualquier tipo de \atacante ?
\end{itemize}

{\noindent \bf Programa y formato de archivos}

El programa debe tomar tres par\'ametros (y en ese orden): el archivo de entrada, el archivo de salida y el m\'etodo a ejecutar, (\emph{0:} Eliminación Gaussiana banda, \emph{1:} Factorización LU explotando la propiedad banda y utilizando forward/back substitution, \emph{2:} Algoritmo de eliminación de \atacante\ simple, \emph{3:} Algoritmo de eliminación de \atacante\ mejorado, usando la fórmula de Sherman–Morrison cuando sea posible).

El archivo de entrada contiene los datos del problema (tama\~no del \objeto, ubicaci\'on, radio y temperatura de las \atacante s) desde un archivo de texto con el siguiente formato:

\begin{verbatim}
      (a)  (b)  (h)  (k)
      (x1) (y1) (r1) (t1)
      (x2) (y2) (r2) (t2)
      ...
      (xk) (yk) (rk) (tk)
\end{verbatim}

En esta descripci\'on, $a\in\real_+$ y $b\in\real_+$ representan el ancho
y largo en metros del \objeto, respectivamente. De acuerdo con la descripci\'on
de la discretizaci\'on del \objeto, $h$ es la longitud de cada intervalo de discretizaci\'on,
obteniendo como resultado una grilla de $n+1\in\nat$ filas y $m+1\in\nat$ columnas. Adem\'as, $k\in\nat$ es la cantidad de \atacante s.
Finalmente, para $i=1,\dots,k$, el par $(x_i,y_i)$ representa la ubicaci\'on de la $i$-\'esima sanguijuela en el \objeto, suponiendo que el punto $(0,0)$  corresponde al extremo inferior izquierdo del mismo, mientras que $r_i\in\real_+$ representa el radio de la sopapa de ataque de la \atacante\ (en metros), y $t_i\in\real$ es la temperatura de dicha sopapa.

El archivo de salida contendr\'a los valores de la temperatura en cada punto de la discretizaci\'on utilizando la informaci\'on original del problema (es decir, antes de aplicar el m\'etodo de remoci\'on de \atacante), y ser\'a utilizado para realizar un testeo parcial de correctitud de la implementaci\'on. El formato del archivo de salida contendr\'a, una por l\'inea, el indicador de cada posici\'on de la grilla $i$, $j$ junto con el correspondiente valor de temperatura. A modo de ejemplo, a continuaci\'on se muestran c\'omo se reportan los valores de temperatura para las posiciones $(3,19)$, $(3,20)$, $(4,0)$ y $(4,1)$. 

\begin{verbatim}
      ...
      3   19  -92.90878
      3   20  -100.00000
      4   0   -100.00000
      4   1   60.03427
      ...
\end{verbatim}

El programa debe ser compilado, ejecutado y testeado utilizando los \emph{scripts} de \emph{python} que acompa\~nan este informe. Estos permiten ejecutar los tests provistos por la c\'atedra, incluyendo la evaluaci\'on de los resultados obtenidos e informando si los mismos son correctos o no. Es requisito que el c\'odigo entregado pase satisfactoriamente los casos de tests provistos para su posterior correcci\'on. Junto con los archivos podr\'an encontrar un archivo \texttt{README} que explica la utilizaci\'on de los mismos.

\vskip 0.5 cm
\hrule
\vskip 0.1 cm

{\bf Sobre la entrega}
\begin{itemize}
\item \textsc{Formato electr\'onico:} Jueves 09 de Abril de 2015, \underline{hasta las 23:59 hs.}, enviando el trabajo
(\texttt{informe} + \texttt{c\'odigo}) a \texttt{metnum.lab@gmail.com}. El \texttt{asunto} del email debe comenzar con el texto \verb|[TP1]| seguido
de la lista de apellidos de los integrantes del grupo. Ejemplo: \texttt{[TP1] Acevedo, Miranda, Montero}
\item \textsc{Formato f\'isico:} Viernes 05 de Abril de 2015, de 17:30 a 18:00 hs.
\end{itemize}

\begin{thebibliography}{1}
	\bibitem{ShermanMorrison} Golub, G. H. and Van Loan, C. F. Matrix Computations, 3rd ed. Baltimore, MD: Johns Hopkins, p. 51, 1996.
\end{thebibliography}

\end{document}